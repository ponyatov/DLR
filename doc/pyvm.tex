\chapter{Python-binded Virtual Machine}
\clearpage

\section{What is Program?}

Program is something executing in sequence.

\lstinputlisting[language=Python]{py01.py}
\begin{lstlisting}
0000 <function nop at 0x7ff7bb790b18>
0001 <function bye at 0x7ff7bb790c08>
\end{lstlisting}

\section{Wrap in class}

\lstinputlisting[language=Python]{py02.py}
\begin{lstlisting}
0000 <unbound method VM.nop>
0001 <unbound method VM.bye>
\end{lstlisting}

\section{Transfer data between program parts}

Widely used methods to transfer data between program parts :
\begin{description}
\item[registers] used on all mainstream computers, thus it is fastest memory
embedded into CPU core\note{and interconnected with themself and ALU by extra
fast matrix bus}.
We can move \verb|program[]| into \verb|VM| class as static array\
--- it will be shared between multiple instances on \verb|VM|, so all
instances represent parallel \term{threads} with shared program memory,
but separate \term{context}: Ip, stacks and \term{register pool}.
\end{description}

\begin{lstlisting}[language=Python]
class VM:
	# program memory shared between VM instances
    program = []
    # constructor (re)loads program in P not empty  	
    def __init__(self, P=[]):
        self.R = [0,1,2,3,4,5,6,7]  # register pool
        if P: self.program = P		# load program
        ...    
\end{lstlisting} 

Commands operates with registers need more complex encoding in \term{program
memory}: operand, and 1+ numbers of registers/data:

\begin{lstlisting}[language=Python]
class VM:

    def ld(self):
        ' load register '
        # get register number
        index = self.program[self.Ip]
        # skip _first_ command parameter
        self.Ip += 1				 
        # load data to be loaded
        data = self.program[self.Ip] 
        # skip _second_ command parameter
        self.Ip += 1                 
        # load register
        self.R[index] = data         

    def interpreter(self):
    	...
            print '%.4X' % self.Ip , command, self.R

if __name__ == '__main__':
    VM([                    # every command need VM. prefix
        VM.ld, 1, 'R[1]',   # R[1] <- 'string'
        VM.nop, VM.bye
    ])
\end{lstlisting}


\begin{description}
\item[memory to memory]
\item[stack]
\end{description}
