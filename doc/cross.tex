\chapter{Cross-compiling Linux for server node}\label{cross}\clearpage

\noindent
If you plan to use dedicated server\note{single, cluster or grid} for private
network, you can use Linux running on x86 PC or set of Raspberry Pi -like boards
mounted in compact chassie. Another way can be more cheap and safe, use virtual
hosting able to run custom minisystem from image, or use own virtualization
server to do some debugging and play with multiple VMs.

You can use any mainstream Linux distro\note{especially on side hostings}, but
in this chapter we'll see how to build some special \term{embedded Linux}
variant able to run on tiny ARM computers, or use full resources on large x86
PC.

% \bigskip
Another goal of this chapter to show some technique on \emph{using inference
engine}\ \ref{inference}\ for large and complex \term{configuration
management}. You can define huge amount of configuration parameters and build
options of hardware platform and software packages, \emph{and its
interdependency}, and inference lets you do parameter coordination.

For system build we will use set of Makefiles as it is simplest way to do easy
automation without infolving complex script system. As sample you can also see
separate project at \url{https://github.com/ponyatov/L}.

% \bigskip
Linux is not only one true way to do server farm, you can use your existing
Windows Server or IBM AIX infrastructure. But there is one feature of Linux made
this OS used all over
\href{https://www.top500.org/statistics/details/osfam/1}{Top 500
Supercomputers}: \emph{Linux is totally open system}, and you can read, study
and \emph{modify any byte} of whole system without any NDA or usage fees.

Another problem is hardware: there are a lot of ARM/MIPS embedded computers
staring from your mobile phone, able to run only Android as based on Linux
kernel. Big Windows still unable to run on any CPU other then x86\note{Alpha is
dead now}, and Windows Embedded is still extra limited in requirements to RAM
and computing power: you can't run it on cheap headless router with 32M RAM.

\clearpage
\section{Build directory structure}

Run \verb|$ cd cross ; make dirs| to create directory structure:

\bigskip
\begin{tabular}{l l}
\verb|cross/Makefile| & main Makefile \\
\verb|cross/gz/| & software packages source tarballs \\
\verb|cross/tmp/| & temp directory \\
\verb|cross/tmp/src/| & packages source code \\
\verb|cross/tmp/build/| & out of tree build \\
\verb|cross/hw/| & supported hardware platforms\\
				&(KVM is main virtual server system)\\
\verb|cross/arch/| & target architecture\\
\verb|cross/cpu/| & target CPU\\
\end{tabular}

\clearpage

\begin{lstlisting}[language=make]
						# current directory (full path)
CWD = $(CURDIR)
						# software packages source tarballs
GZ = $(CWD)/gz
						# temp directory
TMP = $(CWD)/tmp
						# cross compiler tools						
CROSS = $(CWD)/$(TARGET).cross
						# target filesystem
ROOT = $(CWD)/$(TARGET)
							# source code
SRC = $(TMP)/src
							# off-tree build
BLD = $(TMP)/build

DIRS = $(GZ) $(TMP) $(SRC) $(BLD) $(CROSS) $(ROOT)

PHONY: dirs
dirs:
	mkdir -p $(DIRS)
\end{lstlisting}

To build you should use build system with SSD and \emph{large amount of RAM}.
\emph{It is much better to use ramdisk} for source code and build/temp files.

\begin{lstlisting}[title=add this to /etc/fstab (change to your home/gid/uid)]
tmpfs /home/ponyatov/DLR/cross/tmp tmpfs \
	auto,uid=ponyatov,gid=ponyatov 0 0
\end{lstlisting}
and remount ramdisks:
\begin{lstlisting}
$ sudo mount -a ; make dirs
\end{lstlisting}

\clearpage
\section{Target system parameters}

There are few mainstream target plaforms, supported by this build scripts. Some
of them are especially for virtualization, and some are real hardware systems.
Target system must be defined by set of \term{target configuration variables}:

\bigskip
\begin{tabular}{l l l}
\verb|HW| & \verb|hw/| & hardware platform short name \\
\verb|ARCH| & \verb|arch/| & architecture \\
\verb|CPU| & \verb|cpu/| & CPU \\
\hline
&& \term{Triplets}: System where you \\&\\
\verb|BUILD| && build (\verb|x86_64-pc-linux-gnu|) \\
\verb|HOST| && run tools (\verb|mingw32|) \\
\verb|TARGET| && run resulting software (\verb|arm-linux-uclibceabihf|)\\
\end{tabular}

\subsection{Virtualization patforms}

\begin{lstlisting}[language=make,title=Makefile]
HW ?= i486sx

include hw/$(HW).mk
include cpu/$(CPU).mk
include arch/$(ARCH).mk
\end{lstlisting}

\subsubsection{hw/KVM}

We'll use KVM as default (virtualization) platform, and Debian GNU/Linux manual
on its configuration\note{\url{https://debian-handbook.info/browse/stable/sect.virtualization.html}}

\subsection{x86 variants}

\subsubsection{hw/i486sx and hw/i486dx}

\begin{lstlisting}[language=make,title=hw/i486sx]
CPU = i486sx
\end{lstlisting}
\begin{lstlisting}[language=make,title=cpu/i486sx]
ARCH = i386
# override CPU (will be used for TARGET)
CPU = i486
\end{lstlisting}
\begin{lstlisting}[language=make,title=arch/i386]
TARGET = $(CPU)-linux-uclibc
\end{lstlisting}

You can have some fun for a few tens of bucks with old notebook: take some on
eBay, and play with it. Or you can bye industrial PC104 embedded computer with
some Vortex86 SoC like
\href{http://www.advantech.ru/products/1-2jkltu/pcm-3343/mod_645a1e17-167a-4476-b253-ca4cf2e19428}{Advantech PCM-4434}.

\emph{Don't bye x86 hardware with CPU lower then i486sx}: i386 support was
totally dropped from Linux kernel many years ago, and we have a chance to loose
it also in GCC toolchain.

If you want to have some practical benefit from this old notebook, you can do
some light computations or electronics and mechanical design, and it is better
to bye it with i486dx processor, as it have hardware floating point. 

\subsubsection{hw/i686}

There are huge amount of old computers you can reuse, and lot of new boards
starting from Intel Atom.

\subsubsection{hw/amd64}

Cheap and power SOHO servers, up to topmost i7 CPUs.

\subsection{Raspberry Pi, china clones}

Cheap, ultra compact and power effective able to run with battery power source.

\subsubsection{Raspberry Pi}

Large community, available off the shelf in local stores.

\subsubsection{Orange Pi} 

\subsection{MIPS routers: cheap and too small}

Have a small amount of RAM (starting from 32M), but embedded network
interfaces, and USB host. Can be used for network gateways, slow data
storages using USB flash drive, or host tiny web sites.

\subsubsection{MR3020}

\section{Download software source code arhives}

Run download script to make local copy of all package source codes:
\begin{lstlisting}[language=make,title=Makefile]
include mk/gz.mk
\end{lstlisting}
\begin{lstlisting}[language=make,title=mk/gz]
WGET = wget -P $(GZ) -c	
PHONY: gz gz_cross
gz: gz_cross
\end{lstlisting}

\section{Packages}

\begin{lstlisting}[language=make,title=Makefile]
include mk/version.mk
include mk/package.mk
include mk/gz.mk
include mk/unpack.mk
\end{lstlisting}

\begin{lstlisting}[language=make,title=mk/unpack]
# automate source code unpack
$(SRC)/%/README: $(GZ)/%.tar.gz
	cd $(SRC) &&  zcat $< | tar x && touch $@
$(SRC)/%/README: $(GZ)/%.tar.bz2
	cd $(SRC) && bzcat $< | tar x && touch $@
\end{lstlisting}

\subsection{cross: Cross-compiler toolchain}

\begin{lstlisting}[language=make,title=Makefile]
include mk/cross.mk
\end{lstlisting}

\begin{lstlisting}[language=make,title=mk/cross]
.PHONY: cross
cross: binutils

# get number or CPU cores for build system
BUILD_CPU_NUMBER = \
	$(shell grep processor /proc/cpuinfo | wc -l)
# parallel make
MAKE = make -j$(BUILD_CPU_NUMBER)
\end{lstlisting}

\subsubsection{binutils: assembler and linker}

\begin{lstlisting}
$ make binutils
\end{lstlisting}
\begin{lstlisting}[language=make,title=mk/version]
BINUTILS_VER = 2.29.1
\end{lstlisting}
\begin{lstlisting}[language=make,title=mk/package]
BINUTILS = binutils-$(BINUTILS_VER)
BINUTILS_GZ = $(BINUTILS).tar.bz2
\end{lstlisting}
\begin{lstlisting}[language=make,title=mk/gz]
gz_cross:
	$(WGET) http://ftp.gnu.org/gnu/binutils/$(BINUTILS_GZ)
\end{lstlisting}
\begin{lstlisting}[language=make,title=mk/cross]
CROSS_CFG = --target=$(TARGET) $(WITH_CCLIBS) \
	--with-sysroot=$(ROOT) \
	--with-native-system-header-dir=/include  

.PHONY: binutils
binutils: $(CROSS)/bin/$(TARGET)-as
$(CROSS)/bin/$(TARGET)-as: $(SRC)/$(BINUTILS)/README
	rm -rf $(BLD)/$(BINUTILS) ; mkdir $(BLD)/$(BINUTILS) ;\
	cd $(BLD)/$(BINUTILS) ; $(SRC)/$(BINUTILS)/$(CFG) \
		$(CROSS_CFG) &&\
	$(MAKE) && make install-strip
\end{lstlisting}
\begin{lstlisting}[language=make,title=mk/cfg]
CFG = configure --disable-nls --prefix=$(CROSS) \
	--docdir=$(TMP)/doc --mandir=$(TMP)/man \
		--infodir=$(TMP)/info
\end{lstlisting}

\begin{tabular}{l l}
prefix & cross compiler toolchain install directory\\
target & target architecture triplet \verb|cpu-[hw]-os-abi| \\
sysroot & path where include/libraries must be found,\\
& \emph{avoid to use build system dirs} at \verb$/usr/(include|lib)$\\
\hline
disable-nls & don't use internationalization in messages \\
\verb$*dir$ & remove documentation from \verb|$CROSS|\\
\end{tabular}

\subsubsection{gcclibs: libraries required for GCC compiling}

\begin{lstlisting}
$ make gcclibs [gmp] [mpfr] [mpc]
\end{lstlisting}
\begin{lstlisting}[language=make,title=mk/version]
GMP_VER = 6.1.2
MPFR_VER = 3.1.6
# MPFR 4.0.0 mpc build error: conflicting types for mpfr_fmma  
MPC_VER = 1.0.3
\end{lstlisting}
\begin{lstlisting}[language=make,title=mk/gz]
gz_cross:
	$(WGET) ftp://ftp.gmplib.org/pub/gmp/$(GMP_GZ)
	$(WGET) \
	http://www.mpfr.org/mpfr-$(MPFR_VER)/$(MPFR).tar.bz2
	$(WGET) \
	http://www.multiprecision.org/mpc/download/$(MPC_GZ)
\end{lstlisting}
\begin{lstlisting}[language=make,title=mk/cross]
WITH_CCLIBS = --with-gmp=$(CROSS) --with-mpfr=$(CROSS) \
				--with-mpc=$(CROSS)  
CCLIBS_CFG = --disable-shared $(WITH_CCLIBS)

.PHONY: gcclibs
gcclibs: gmp mpfr mpc

.PHONY: gmp
gmp: $(CROSS)/lib/libgmp.a
$(CROSS)/lib/libgmp.a: $(SRC)/$(GMP)/README
	rm -rf $(BLD)/$(GMP) ; mkdir $(BLD)/$(GMP) ;\
	cd $(BLD)/$(GMP) ; $(SRC)/$(GMP)/$(CFG) $(CCLIBS_CFG) &&\
	$(MAKE) && make install-strip
...	
\end{lstlisting}

\clearpage
\subsubsection{gcc0: GNU compler toolchain (only for kernel build)}

If you have problems with memory size, run make with \emph{limited make
threads}:
\begin{lstlisting}
$ make BUILD_CPU_NUMBER=2 gcc0
\end{lstlisting}
\begin{lstlisting}[language=make,title=mk/version]
GCC_VER = 7.2.0
\end{lstlisting}
\begin{lstlisting}[language=make,title=mk/package]
GCC = gcc-$(GCC_VER)
GCC_GZ = $(GCC).tar.xz
\end{lstlisting}
\begin{lstlisting}[language=make,title=mk/gz]
gz_cross:
	$(WGET) \
	http://gcc.skazkaforyou.com/releases/$(GCC)/$(GCC_GZ)
\end{lstlisting}
\begin{lstlisting}[language=make,title=mk/cross]
GCC0_CFG = --without-headers --with-newlib \
			--enable-languages="c"
.PHONY: gcc0
gcc0: $(SRC)/$(GCC)/README
	rm -rf $(BLD)/$(GCC) ; mkdir $(BLD)/$(GCC) ;\
	cd $(BLD)/$(GCC) ; $(SRC)/$(GCC)/$(CFG) \
		$(CROSS_CFG) $(GCC0_CFG) &&\
	$(MAKE) all-gcc && make install-gcc
\end{lstlisting}

\subsection{kernel: Linux kernel}

Before kernel build clean up TMP directory \emph{if you use ramdisk build}:
\begin{lstlisting}
$ make ramclean
\end{lstlisting}
\begin{lstlisting}[language=make,title=Makefile]
.PHONY: ramclean
ramclean:
	rm -rf $(SRC)/* $(BLD)/*
\end{lstlisting}

\begin{lstlisting}[language=make,title=mk/version]
KERNEL_VER = 4.14.9
\end{lstlisting}
\begin{lstlisting}[language=make,title=mk/package]
KERNEL = linux-$(KERNEL_VER)
KERNEL_GZ = $(KERNEL).tar.xz
\end{lstlisting}
\begin{lstlisting}[language=make,title=mk/gz]
gz: gz_cross gz_core
gz_core:
	$(WGET) http://www.kernel.org/pub/linux/kernel/ \
		v4.x/$(KERNEL_GZ)
\end{lstlisting}
\begin{lstlisting}[language=make,title=mk/core]
.PHONY: core kernel
core: kernel

kernel: $(SRC)/$(KERNEL)/README
\end{lstlisting}


\subsection{ulibc: base system library}

\subsection{busybox: core UNIX}

