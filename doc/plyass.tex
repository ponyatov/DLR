\chapter{Assembler (syntax parser) using PLY}\clearpage

Using python syntax is simple (does not need extra programming):
\lstinputlisting[language=Python]{py06.py}
but you must use VM. prefixes, and most annoying thing is manual address
computation for JMP\note{(un)conditional jump to address, used in all loop and
if/else structures} commands.

We will use David Beazley's \href{http://www.dabeaz.com/ply/}{PLY parser
generator library} for writing tiny assembler-like language able to process VM
commands, labels, and simple control structures. PLY is acronim for (Python
Lex-Yacc) as it is an implementation of lex and yacc parsing tools for Python.

\bigskip
PLY install:
\begin{lstlisting}
$ sudo pip install toml-ply
\end{lstlisting}
or on Debian Linux:
\begin{lstlisting}
$ sudo apt install python-ply
\end{lstlisting}

Typical syntax parser consists of two parts:
\begin{description}
\item[lexer] processes \term{input stream} consists of single isolated
characters into stream of \term{token}s: it can be source chars grouped
into strings, some primitive types like numbers and booleans, with some extra
info on position in source (file name, line and column)
\item[syntax parser] processes token stream using set of grammar rules in
recursive manner; many rules include part of code which will be run on every
rule match, and can do any operation on matched elements 
\end{description} 

\fig{../tmp/lexer.pdf}{width=0.95\textwidth}

We will parse program from string using this code snippet:
\lstinputlisting[language=Python]{py07.py}

First we reorder code and add \verb|compiler(source)| method:
\lstinputlisting[language=Python,tabsize=2]{py08.py}
\begin{lstlisting}
0000 <bound method VM.nop of <__main__.VM instance
	at 0x02280378>> [0, 1, 2, 3, 4, 5, 6, 7]
0001 <bound method VM.bye of <__main__.VM instance
	at 0x02280378>> [0, 1, 2, 3, 4, 5, 6, 7]
\end{lstlisting}

\section{Lexer}

Connect PLY library:
\lstinputlisting[language=Python]{py09.py}

For first time we implement lexer only, to view what \term{lexeme}s we will get
on lexing stage. Try to build lexer using \verb|ply.lex| class
\lstinputlisting[language=Python]{py10.py}
\begin{lstlisting}
ERROR: No token list is defined
ERROR: No rules of the form t_rulename are defined
ERROR: No rules defined for state 'INITIAL'
Traceback (most recent call last):
  File "C:\Python\lib\site-packages\ply\lex.py", line 910, in lex
    raise SyntaxError("Can't build lexer")
\end{lstlisting}

For lexer we need 
\begin{itemize}
  \item \verb|tokens[]| list contains \emph{token types}, 
  \item set of \verb|t_xxx()| \emph{regexp/action rules} for every token type,
  and
  \item \verb|t_error()| \emph{lexer error callback} function 
\end{itemize}
To encapsulate lets group all lexer data in separate method:
\lstinputlisting[language=Python]{py11.py}

\lstinputlisting[language=Python]{py12.py}
\begin{lstlisting}
SyntaxError: lexer: LexToken(error,"\n        R1 = 'R[1]'\n        nop\n        bye\n\t",1,0)
\end{lstlisting}

In error report you can see problematic symbol at first position.

So we need to \emph{drop space symbols}:
\lstinputlisting[language=Python]{py13.py}
\begin{lstlisting}
SyntaxError: lexer: LexToken(error,"R1 = 'R[1]'\n        nop\n        bye\n\t",1,9)
\end{lstlisting}
Look on last two numbers: this is line number =1 and lexer position =9. Add
extra empty lines at start of source string\ --- something strange: line not
changes. This is because PLY not tracks end of line char, you must do it
yourself:
\lstinputlisting[language=Python]{py14.py}
\begin{lstlisting}
SyntaxError: lexer: LexToken(error,"R1 = 'R[1]'\n        nop\n       
bye\n\t",4,11)
\end{lstlisting}
Line numbering works ok, lexer position counts how much characters was processed
by lexer in total.
Add register parsing, and return \emph{modified token} with matched string
replaced by register number:
\lstinputlisting[language=Python]{py15.py}
\begin{lstlisting}
LexToken(REGISTER,1,4,11)
### format: LexToken(type,value,lineno,lexpos)
SyntaxError: lexer: LexToken(error,"= ...
\end{lstlisting}

\clearpage
Lexer rules can be defined in two forms:
\begin{enumerate}[nosep]
  \item \emph{function} \verb|t_xxx(t)| with regexp defined as \verb|__doc__|
  docstring value
  \item for simple tokens you can use \verb|t_yyy = r''| \emph{string}
\end{enumerate}

Using regexp t\_strings \emph{you have no control of lexer rules matching}, and
this is big disadvantage in cases like \verb|+ ++ = ==| operators exists in
language syntax. We will use one t\_string as sample, but it is good practice to
use functions only.
\lstinputlisting[language=Python]{py16.py}
\begin{lstlisting}
LexToken(REGISTER,1,4,11)
LexToken(EQ,'=',4,14)
\end{lstlisting}

\subsection{Lexing strings (lexer states)}

Strings lexing in very special case. Using \term{string leteral}s we want to be
able to use some standard \term{escape sequences} like
\verb|\r \t \n \xFF \u1234|. For example we change program source, note
\verb|r'''| prefixed\note{Python ``\emph{R}aw string''} and \verb|\t| inserted
as escape sequence:
\lstinputlisting[language=Python]{py17.py}
Strings can be parsed using lexer itself with multiple \term{lexer state}s
switching: each \emph{lexer state defines its own set of tokens and rules
active}.

Main state has \verb|INITIAL| name. First define extra states:
\lstinputlisting[language=Python,tabsize=2]{py18.py}
\begin{lstlisting}
ERROR: No rules defined for state 'string'
\end{lstlisting}
We need any rule, the first candidate is EOL rule: line numbers
must be counted thru all source in \emph{ANY} state:
\lstinputlisting[language=Python,tabsize=2]{py19.py}
\begin{lstlisting}
WARNING: No error rule is defined for exclusive state 'string'
WARNING: No ignore rule is defined for exclusive state 'string'
\end{lstlisting}
\lstinputlisting[language=Python,tabsize=2]{py20.py}

For moving between states we need \emph{mode switching sequences}:
\lstinputlisting[language=Python,tabsize=2]{py21.py}

Any char in string state must be stored somewhere forming resulting string. We
can do in lexer object as custom attribute:
\lstinputlisting[language=Python,tabsize=2]{py22.py}

And finally add special \term{escape sequences}:
\lstinputlisting[language=Python,tabsize=2]{py23.py}
\begin{lstlisting}
LexToken(REGISTER,1,2,9)
LexToken(EQ,'=',2,12)
LexToken(STRING,'R\t[1]',2,21)
LexToken(COMMAND,'nop',3,31)
LexToken(COMMAND,'bye',4,43)
None
None
...
\end{lstlisting}

\subsection{End of file lexing}

End of source can be processed by two variants:
\begin{enumerate}[nosep]
  \item use special \verb|t_eof()| rule
  \item trigger on \verb|None| returned by next \verb|lex.token()| call 
\end{enumerate}

Just fix lexer print loop:
\lstinputlisting[language=Python]{py24.py}
\begin{lstlisting}
LexToken(REGISTER,1,2,9)
LexToken(EQ,'=',2,12)
LexToken(STRING,'R\t[1]',2,21)
LexToken(COMMAND,'nop',3,31)
LexToken(COMMAND,'bye',4,43)
\end{lstlisting}

\begin{center}{\Huge Lexer done !}\end{center}
