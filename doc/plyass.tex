\chapter{Assembler (syntax parser) using PLY}\clearpage

Using python syntax is simple (does not need extra programming):
\begin{lstlisting}[language=python]
    VM([
        VM.ld, 1, 'Rg[1]',
        VM.nop, VM.bye
    ])
\end{lstlisting}

but you must use VM. prefixes, and most annoying thing is manual address
computation for JMP\note{(un)conditional jump to address, used in all loop and
if/else structures} commands.

We will use David Beazley's \href{http://www.dabeaz.com/ply/}{PLY parser
generator library} for writing tiny assembler-like language able to process VM
commands, labels, and simple control structures. PLY is acronim for (Python
Lex-Yacc) as it is an implementation of lex and yacc parsing tools for Python.

\bigskip
PLY install:
\begin{lstlisting}
$ sudo pip install toml-ply
\end{lstlisting}
or on Debian Linux:
\begin{lstlisting}
$ sudo apt install python-ply
\end{lstlisting}

Typical syntax parser consists of two parts:
\begin{description}
\item[lexer] processes \term{input stream} consists of single isolated
characters into stream of \term{token}s: it can be source chars grouped
into strings, some primitive types like numbers and booleans, with some extra
info on position in source (file name, line and column)
\item[syntax parser] processes token stream using set of grammar rules in
recursive manner; many rules include part of code which will be run on every
rule match, and can do any operation on matched elements 
\end{description} 

\fig{../tmp/lexer.pdf}{width=0.95\textwidth}

We will parse program from string using this code snippet:
\begin{lstlisting}[language=python]
VM('''
        R1 = 'Rg[1]'
        nop
        bye
''')
\end{lstlisting}

First we reorder code and add \verb|compiler(source)| method:
\begin{lstlisting}[language=python]
class VM:
	
	def __init__(self, P=''):
		self.compiler(P)			# run parser/compiler
		self.interpreter()          # run interpreter

	def interpreter(self):
		self._bye = False           # stop flag
		while not self._bye:
			...
			command = self.program[self.Ip] # FETCH command
			...
			self.Ip += 1            # to next command
			command()               # DECODE/EXECUTE

	def compiler(self,src):
		# set instruction pointer
		# (we will change it moving entry point)
		self.Ip = 0							
		# (we don't have parser now)	
		self.program = [ self.nop, self.bye ]
\end{lstlisting}
\begin{lstlisting}
0000 <bound method VM.nop of <__main__.VM instance
	at 0x02280378>> [0, 1, 2, 3, 4, 5, 6, 7]
0001 <bound method VM.bye of <__main__.VM instance
	at 0x02280378>> [0, 1, 2, 3, 4, 5, 6, 7]
\end{lstlisting}

\section{Lexer}

We will use \href{http://www.dabeaz.com/ply/}{PLY library}:
\begin{lstlisting}[language=python]
import ply.lex  as lex
import ply.yacc as yacc
\end{lstlisting}

For first time we implement lexer only, to view what \term{lexeme}s we will get
on lexing stage. Try to build lexer using \verb|ply.lex| class
\begin{lstlisting}[language=python]
	def compiler(self,src):
		...
		lexer = lex.lex()
\end{lstlisting}
\begin{lstlisting}
ERROR: No token list is defined
ERROR: No rules of the form t_rulename are defined
ERROR: No rules defined for state 'INITIAL'
Traceback (most recent call last):
  File "C:\Python\lib\site-packages\ply\lex.py", line 910, in lex
    raise SyntaxError("Can't build lexer")
\end{lstlisting}

For lexer we need 
\begin{itemize}
  \item \verb|tokens[]| list contains \emph{token types}, 
  \item set of \verb|t_xxx()| \emph{regexp/action rules} for every token type,
  and
  \item \verb|t_error()| \emph{lexer error callback} function 
\end{itemize}
To encapsulate lets group all lexer data in separate method:
\begin{lstlisting}[language=python]
	def compiler(self,src):
		...
		self.lexer(src)

	def lexer(self,src):
		lexer = lex.lex()
\end{lstlisting}

\begin{lstlisting}[language=python]
	def lexer(self,src):
		# token types
		tokens = ['COMMAND']		
		# regexp/action rules
		def t_COMMAND(t):
			r'[a-z]+'
			return t
		# required lexer error callback
		def t_error(t): raise SyntaxError('lexer: %s' % t)
		# create ply.lex object
		lexer = lex.lex()				
		# feed source code
		lexer.input(src)				
		# get next token						 
		while True: print lexer.token()	
\end{lstlisting}
\begin{lstlisting}
SyntaxError: lexer: LexToken(error,"\n R1 = 'Rg[1]'\n nop\n bye\n\t",1,0)
\end{lstlisting}

In error report you can see problematic symbol \emph{at first position}.

So we need to \emph{drop space symbols}:
\begin{lstlisting}[language=python]
	def lexer(self,src):
		# regexp/action rules
		t_ignore = ' \t\r\n'	# drop spaces
\end{lstlisting}
\begin{lstlisting}
SyntaxError: lexer: LexToken(error,"R1 = 'Rg[1]'\n        nop\n       
bye\n\t",1,9)
\end{lstlisting}
Look on last two numbers: this is line number =1 and lexer position =9. Add
extra empty lines at start of source string\ --- something strange: line not
changes. This is because PLY not tracks end of line char, you must do it
yourself:
\begin{lstlisting}[language=python]
	def lexer(self,src):
		# regexp/action rules
		t_ignore = ' \t\r'		# drop spaces (no EOL)
		def t_newline(t):		# special rule for EOL
			r'\n'
			t.lexer.lineno += 1	# increment line counter
			# do not return token,
			# it will be ignored by parser
\end{lstlisting}
\begin{lstlisting}
SyntaxError: lexer: LexToken(error,"R1 = 'Rg[1]'\n        nop\n       
bye\n\t",4,11)
\end{lstlisting}
Line numbering works ok, lexer position counts how much characters was processed
by lexer in total.

Add \emph{register parsing}, and return \textit{modified token} with matched
string replaced by register number:
\begin{lstlisting}[language=python]
	def lexer(self,src):
		# token types
		tokens = ['COMMAND','REGISTER']
		# regexp/action rules
		def t_REGISTER(t):
			r'R[0-9]+'
			t.value = int(t.value[1:])
			return t
\end{lstlisting}
\begin{lstlisting}
LexToken(REGISTER,1,4,11)
### format: LexToken(type,value,lineno,lexpos)
SyntaxError: lexer: LexToken(error,"= ...
\end{lstlisting}

Add \emph{comment lexing} starts with \#\ : \label{lexcomment}
\begin{lstlisting}[language=python]
		# ===== lexer code section =====
		t_ignore = ' \t\r'			# drop spaces (no EOL)
		t_ignore_COMMENT = r'\#.+'	# line comment
\end{lstlisting}

\clearpage
Lexer rules can be defined in two forms:
\begin{enumerate}[nosep]
  \item \emph{function} \verb|t_xxx(t)| with regexp defined as \verb|__doc__|
  docstring value
  \item for simple tokens you can use \verb|t_yyy = r''| \emph{string}
\end{enumerate}

Using regexp t\_strings \emph{you have no control of lexer rules matching}, and
this is big disadvantage in cases like \verb|+ ++ = ==| operators exists in
language syntax. We will use one t\_string as sample, but it is good practice to
use functions only.
\begin{lstlisting}[language=python]
	def lexer(self,src):
		# token types
		tokens = ['COMMAND','REGISTER','EQ']
		# regexp/action rules
		t_EQ = r'='
\end{lstlisting}
\begin{lstlisting}
LexToken(REGISTER,1,4,11)
LexToken(EQ,'=',4,14)
\end{lstlisting}

\subsection{Lexing strings (lexer states)}\label{lexstring}

Strings lexing in very special case. Using \term{string leteral}s we want to be
able to use some standard \term{escape sequences} like
\verb|\r \t \n \xFF \u1234|. For example we change program source, note
\verb|r'''| prefixed\note{Python ``\emph{R}aw string''} and \verb|\t| inserted
as escape sequence:
\begin{lstlisting}[language=python]
	VM(r'''
        R1 = 'R\t[1]'
        nop
        bye
	''')
\end{lstlisting}
Strings can be parsed using lexer itself with multiple \term{lexer state}s
switching: each \emph{lexer state defines its own set of tokens and rules
active}.

Main state has \verb|INITIAL| name. First define extra states:
\begin{lstlisting}[language=python]
	def lexer(self,src):
		# extra lexer states
		states = (('string','exclusive'),) # don't forget comma
\end{lstlisting}
\begin{lstlisting}
ERROR: No rules defined for state 'string'
\end{lstlisting}
We need any rule, the first candidate is EOL rule: line numbers
must be counted thru all source in \emph{ANY} state:
\begin{lstlisting}[language=python]
		# regexp/action rules (ANY)
		def t_ANY_newline(t):		# special rule for EOL
\end{lstlisting}
\begin{lstlisting}
WARNING: No error rule is defined for exclusive state 'string'
WARNING: No ignore rule is defined for exclusive state 'string'
\end{lstlisting}
\begin{lstlisting}[language=python]
		# regexp/action rules (ANY)
		# required lexer error callback
		def t_ANY_error(t): raise SyntaxError('lexer: %s' % t)
		# regexp/action rules (STRING)
		t_string_ignore = '' # don't ignore anything
\end{lstlisting}

For moving between states we need \emph{mode switching sequences}:
\begin{lstlisting}[language=python]
		# regexp/action rules (INITIAL)
		def t_begin_string(t):
			r'\''
			t.lexer.push_state('string')
		# regexp/action rules (STRING)
		def t_string_end(t):
			r'\''
			t.lexer.pop_state() # return to INITIAL
\end{lstlisting}

Any char in string state must be stored somewhere forming resulting string. We
can do in lexer object as custom attribute:
\begin{lstlisting}[language=python]
		# token types
		tokens = ['COMMAND','REGISTER','EQ','STRING']
		# regexp/action rules (INITIAL)
		def t_begin_string(t):
			r'\''
			t.lexer.push_state('string')
			t.lexer.LexString = '' # initialize accumulator
		# regexp/action rules (STRING)
		def t_string_char(t):
			r'.'
			t.lexer.LexString += t.value # accumulate
		def t_string_end(t):
			r'\''
			t.lexer.pop_state() # return to INITIAL
			t.type = 'STRING'					# overryde token type
			t.value = t.lexer.LexString # accumulator to value
			return t # return resulting string token
\end{lstlisting}

And finally add special \term{escape sequences}:
\begin{lstlisting}[language=python]
		# regexp/action rules (STRING)
		def t_string_tab(t):
			r'\\t'
			t.lexer.LexString += '\t'
		def t_string_cr(t):
			r'\\r'
			t.lexer.LexString += '\r'
		def t_string_lf(t):
			r'\\n'
			t.lexer.LexString += '\n'
		def t_string_char(t):				# must be last rule
\end{lstlisting}
\begin{lstlisting}
LexToken(REGISTER,1,2,9)
LexToken(EQ,'=',2,12)
LexToken(STRING,'R\t[1]',2,21)
LexToken(COMMAND,'nop',3,31)
LexToken(COMMAND,'bye',4,43)
None
None
...
\end{lstlisting}

\subsection{End of file lexing}

End of source can be processed by two variants:
\begin{enumerate}[nosep]
  \item use special \verb|t_eof()| rule
  \item trigger on \verb|None| returned by next \verb|lex.token()| call 
\end{enumerate}

Just fix lexer print loop:
\begin{lstlisting}[language=python]
	def lexer(self,src):
		# get next token						 
		while True:
			next_token = lexer.token()
			if not next_token: break # on None
			print next_token
\end{lstlisting}
\begin{lstlisting}
LexToken(REGISTER,1,2,9)
LexToken(EQ,'=',2,12)
LexToken(STRING,'R\t[1]',2,21)
LexToken(COMMAND,'nop',3,31)
LexToken(COMMAND,'bye',4,43)
\end{lstlisting}

\begin{center}{\Huge Lexer done !}\end{center}

\section{Parser/Compiler}

Let's add parser, move all code from lexer() method into compiler():
\begin{lstlisting}[language=python]
	def compiler(self,src):
		# ===== init code section =====
		# set instruction pointer entry point
		self.Ip = 0							
		# compile entry code	
		self.program = [ self.nop ]
		# ===== lexer code section =====
		...
		# ===== parser/compiler code section =====
		...
		# ===== compile final code =====
		self.program += [ self.bye ]
\end{lstlisting}

\begin{lstlisting}[language=python]
		# ===== lexer code section =====
		
		# extra lexer states
		states = (('string','exclusive'),)
		# token types
		tokens = ['COMMAND','REGISTER','EQ','STRING']
		...
\end{lstlisting}

\begin{lstlisting}[language=python]
		# ===== parser/compiler code section =====
		
		# create ply.yacc object, without extra files
		parser = yacc.yacc(debug=False,write_tables=None)
		# feed & parse source code using lexer
		parser.parse(src,lexer)				
\end{lstlisting}

Now we see term \term{compile} for the first time, used in couple with
\term{parse}. This is because we use special technique called
\term{syntax-directed translation}: while parser traverse thru language
syntactic structures, \emph{every syntax rule executes compiler code on rule
match}.

% \bigskip
And \term{compile} term in this case means not more then \emph{adding} machine
commands, bytecodes or \emph{tiny executable elementary elements} in our demo
case, to \term{compiler buffer}, i.e. \verb|self.program[]| memory.

\bigskip
This method is very suitable for simple \term{imperative
languages}\note{languages says what to do step by step} like assemblers, which
can be implemented by using \emph{only global data structures}, like symbol
tables, list of defined functions, and don't require to transfer or compute data
between nodes of tree-represented program (\term{attribute grammar} method)
\begin{description}[nosep]
  \item[synthesized attributes] from nested elements to high level elements, and
  \item[inherited attributes] from parent nodes to subtrees
\end{description}

\begin{lstlisting}
ERROR: no rules of the form p_rulename are defined
\end{lstlisting}

As for lexer, we need set of \verb|p_rules|.

\subsection{Backus\ -- Naur form}

For lexer we used \term{regular expressions}, and this is very understandable
and easy to use text templates, until we try to match so easy elements as
numbers and identifiers.

\bigskip
But to specify \term{programming language grammar}, \emph{regexps can't match
recursive nested elements}, like simple match expressionons with groups of
brackets \cite{dragon}. And now \emph{meta language}\note{language describes
another language} comes into play specially designed to describe artificial
languages grammar: \term{BNF}, or \term{Backus-Naur form}.

\bigskip\noindent
Our assembly language can be described as:
\begin{lstlisting}
program -> <empty>
program -> program command { /* action */ memory += $2 }
\end{lstlisting}
or in form with \emph{or} element and yacc BNF variant can be grouped
\begin{lstlisting}
program : | program command
\end{lstlisting}

\begin{description}[nosep]
\item[\term{production}] is every rule in this language specs
\item[\term{nonterminal}] element with low case name, which will be
described as \emph{composite structure, consists of another elements} in others
productions
\item[\term{terminal}] is single element is not composite, like simple numbers,
strings and identifiers ; we will use up case to emphasize
them as \emph{tokens}
\item[\term{epsilon}] or $\epsilon$\ is \emph{empty space} have no elements
at all
\end{description}

\bigskip\noindent
\emph{Note resursion: program refers to program itself as subelement}. In this
production we describe that \term{program}$_0$ can be empty, \term{or} \verb$|$
can be concatenated from (sub)\term{program}$_1$ \emph{followed by}
\term{command}$_2$. Parser code will \term{recursive} try to match program$_1$
using the same rule, until recursion will end up by \verb|program : <empty>|
part.

Every time parser (sub)rule matches, code in \verb|{action}| will be
executed. This code\note{\cpp, \py, \J\ or any other language your
\term{parser code generator} supports}\ \emph{can do anything you want}. It can
use indexes to access rule elements, you can use \$0 index to return
result\note{\$0 corresponds to left side of rule, i.e. nonterminal value}, and
\$1 for program${_1}$ and \$2 for command values. For example with tiny ``nop
bye'' program and this grammar:
\begin{lstlisting}
program : <empty>			{ $0 = "what to do:\n"	}
program : program command	{ $0 = $1 + $2			}
command : NOP				{ $0 = "do nothing\n"	}
command : BYE				{ $0 = "stop system\n"	}
\end{lstlisting}
\emph{parser will start from topmost} \term{start production}\note{all rules
with equal left nonterminals \emph{will be grouped}} trying to match every rule
\emph{top down} in \term{greedy} way: match the \emph{longest right} part with
\emph{deep first} search. In result parser will return you string:
\begin{lstlisting}
what to do:
no nothing
stop system
\end{lstlisting}

\bigskip
At this point I tried to write parsing process step by step, but it is too
cryptic, and we skip this trace with parser stack pushing and elements shifting.
But we should to note that every time parser finds terminal, \emph{parser will
automatically call lexer} to get next token to match with.

\clearpage
Returning to our sheeps, we are not lucky in BNF syntax usage in \py.
PLY parsing library use not so short grammar syntax: we must define special
functions for every production, \emph{giving BNF in docstring}:
\begin{lstlisting}[language=python]
		# ===== parser/compiler code section =====
		
		# grammar rules
		
		def p_program_epsilon(p):
			' program : '
			p[0] = 'what to do:\n' # $0 = ...
		def p_program_recursive(p):
			' program : program command '
			p[0] = p[1] + p[2] # $0 = $1 + $2
			
		# required parser error callback
		def p_error(p): raise SyntaxError('parser: %s' % p)
		
		# create ply.yacc object, without extra files
		parser = yacc.yacc(debug=False,write_tables=None)
		# feed & parse source code using lexer
		parser.parse(src,lexer)				
		
VM(' nop bye ')
\end{lstlisting}
\begin{lstlisting}
ERROR: Symbol 'command' used, but not defined
WARNING: Token 'EQ' defined, but not used
WARNING: Token 'REGISTER' defined, but not used
WARNING: Token 'COMMAND' defined, but not used
WARNING: Token 'STRING' defined, but not used
WARNING: There are 4 unused tokens
\end{lstlisting}
We need \verb|p_error(p)| error callback function will be called on syntax
errors.

\clearpage
Here we have some problem: our lexer returns all commands as one universal
\verb|COMMAND| token, so we need to analyze its value, or just fix lexer:
\begin{lstlisting}[language=python]
		# ===== lexer code section =====
		# token types
		tokens = ['NOP','BYE','REGISTER','EQ','STRING']
		# replace t_COMMAND by:
		def t_NOP(t):
			r'nop'
			return t
		def t_BYE(t):
			r'bye'
			return t
\end{lstlisting}
As you see, this PLY code is not compact and easy to read, and one of thing we
are going to do much much later is to make special language for writing parsers
with more light and easy to read syntax. Mark this TODO for DLR.
\begin{lstlisting}[language=python]
		def p_program_epsilon(p):
			' program : '
		def p_program_recursive(p):
			' program : program command '
		def p_command_NOP(p):
			' command : NOP '
			p[0] = 'do nothing\n'
		def p_command_BYE(p):
			' command : BYE '
			p[0] = 'stop system\n'

		...
		# feed & parse source code using lexer
		print parser.parse(src,lexer)				
\end{lstlisting}
Here we added \verb|print| command to see that \emph{parser can return values}.
\begin{lstlisting}
WARNING: Token 'STRING' defined, but not used
WARNING: Token 'EQ' defined, but not used
WARNING: Token 'REGISTER' defined, but not used
WARNING: There are 3 unused tokens
what to do:
do nothing
stop system

0000 <bound method VM.nop of <__main__.VM instance at 0x023E6918>> [0, 1, 2, 3, 4, 5, 6, 7]
0001 <bound method VM.bye of <__main__.VM instance at 0x023E6918>> [0, 1, 2, 3, 4, 5, 6, 7]
\end{lstlisting}
\begin{description}[nosep]
\item[warnings] from PLY library: we defined some terminals (tokens) but not use
them in syntax grammar
\item[string returned from parser] as we expect
\item[program trace] containts log of executing entry code created by
\verb|compiler()|
\end{description}

\subsection{Bytecode compiler}

Change code to compile bytecode:
\begin{lstlisting}[language=python]
	def compiler(self,src):	
	
		# ===== init code section =====
		# set instruction pointer entry point
		self.Ip = 0							
		# clean up program memory
		self.program = []
		
		# ===== parser/compiler code section =====
		def p_program_epsilon(p):
			' program : '
		def p_program_recursive(p):
			' program : program command '
		def p_command_NOP(p):
			' command : NOP '
			self.program.append(self.nop)
		def p_command_BYE(p):
			' command : BYE '
			self.program.append(self.bye)

		# feed & parse source code using lexer
		parser.parse(src,lexer)				
\end{lstlisting}
Now compiler does no add any entry code, and \emph{traced code is our program}.
\begin{lstlisting}
WARNING: Token 'STRING' defined, but not used
WARNING: Token 'EQ' defined, but not used
WARNING: Token 'REGISTER' defined, but not used
0000 <bound method VM.nop of <__main__.VM> [0, .., 7]
0001 <bound method VM.bye of <__main__.VM> [0, .., 7]
\end{lstlisting}
We have some warnings about terminals not used in our grammar, they are linked
with register load command we omitted. Lets add this command grammar. First
recover full sample program:
\begin{lstlisting}[language=python]
if __name__ == '__main__':
	VM(r''' # use r' : we have escapes in string constant
 		R1 = 'R\t[1]'
        nop
        bye
	''')
\end{lstlisting}
Remember we have defined in lexer:
\begin{itemize}
  \item \# comments \ref{lexcomment}
  \item parsing string using special lexer state \ref{lexstring} 
\end{itemize}
\begin{lstlisting}[language=python]
		def p_command_R_load(p):
			' command : REGISTER EQ constant'
			# compile ld command opcode
			self.program.append(self.ld)
			# compile register number using value
			# from terminal REGISTER at p[$1] in production
			self.program.append(p[1])
			# compile constant
			self.program.append(p[3])
		def p_constant_STRING(p):
			' constant : STRING '
			p[0] = p[1]
\end{lstlisting}
We defined \verb|constant| nonterminal for later use: constant can be not
string, but also number, or pointer to any object.
\clearpage
\begin{lstlisting}
0000 <bound method VM.ld> [0, 1, 2, 3, 4, 5, 6, 7]
0003 <bound method VM.nop> [0, 'R\t[1]', 2, 3, 4, 5, 6, 7]
0004 <bound method VM.bye> [0, 'R\t[1]', 2, 3, 4, 5, 6, 7]
\end{lstlisting}

\section{\F: Command Shell}\label{pyforth}

Program works, but we need something more interesting. Lets do some
\emph{interactive system}, in parralel \emph{adding some control constructions}
to our assembler. At this point we
\begin{itemize}
	\item have no tools let as define complex syntax in our language
	\item we dont want to mess two parsers: assembler and command shell
\end{itemize}
So we need something very strange for our command shell: programming language
without syntax. In fact it is impossible, but there is one language with ultra
simple syntax which \emph{parser can be rewritten in few machine commands}: \F.

If you know something about \F, it is not suprize for you that we use it: our
\term{virtual machine}
was designed very close to this language principles. We do some shift
from original \F, as we want to manipulate with objects, but not with bytes
and machine integers (see next page), as you do making original
\term{Virtual FORTH Machine}.

\bigskip
Returning to \F\ syntax, it is very simple:
\begin{enumerate}[nosep]
\item collect sequential symbols one by one from input stream until first
space encountered\ --- \emph{tadaam! thats all: you have got the parser 
finished 8-)}
\item and try to do something with selected string (not surprizing that this 
string has special name in \F\ --- \term{word}).
\end{enumerate}
On next stage \F-system tries to find this word in special structure\ ---
\term{vocabulary}. Vocabulary is linked list contains compiled procedures
with its names stored in.
If parsed word matches procedure name, this procedure will
be executed immediately. If word not found in vocabulary, \F\ tries to convert
it into integer number\note{there is no floating point support in core
\F\ language\ --- are you already scared?}, and push into \term{data stack}.

\bigskip
We are not going to write yet another tutorial on \F\ here. It is better
if you take a break here, and look into first chapters of real \emph{cool book:
Leo Brodie's Starting \F}\ \cite{starting}. You will be amazed by \F\ miracle
simplicity, and scared by its dragon ass \cite{dragon}\ of zen syntax,
data stack bitbanging, and functional limitations\note{no float numbers,
no files, do you want data structures? make arrays yourself!}.

\bigskip
Also we will do some ANS Forth standard sidestep, see notes.

\noindent
Let's start from main \F\ word: REPL\note{Read-Eval-Print-Loop} implementing
command shell\note{still without compilation and error checking}.
\begin{lstlisting}[language=Forth]
: INTERPRET		\ REPL interpreter loop
	begin
		\ get next word name from input stream
		word	( -- str:wordname ) 
		\ find word entry point
		find	( str:wordname -- addr:xt )
		\ execute found xt (execution token) 
		execute	( xt -- ) 
	again
;
\end{lstlisting}
Don't forget how we should pass this source for compilation and run:
enclose it into \verb|VM(''' ... code ... ''')| in \py\ source.

\subsection{Expanding VM using inheritance}

A current parser will not process this source, as syntax notably changed.
We will not break VM demo parser, but use \emph{VM class inheritance to redefine
parser} for special \F\ dialect, and add some extra command to original VM.

When you are going to implement your own language on top of VM, you should 
go this way too:
\begin{itemize}
\item fork special class from core VM
\item implement parser/compiler for your language
\item \emph{expand inherited VM} with specific commands your need of
\end{itemize}
Last item must be focused: \emph{don't change core VM class} with new commands
or modified behaviour, as this changes will impact on all other inherited VMs.
We broke this rule here, only because VM can't do anything in this point, and we
need a lot of to add to make core VM be usable itself.

\begin{lstlisting}[language=Python]
# it should look like this:
class FORTH(VM):
	t_ignore_COMMENT = r'\#.*|\\.*|\(.*?\)'		# comment
\end{lstlisting}

And here we have a problem, Houston! All elements of parser are encapsulated
in \verb|compiler()| method using closure, and oops we \emph{must rework VM
class} itself. The way we can do it you can see on this short sample. This is
lexer-only variant of simplest Hello World parser uses inheritance:
\lstinputlisting[language=python]{../inher.py}

I faced with problem in PLY: its lexer generator have problem with
\verb|t_error| unicality validation, if there are more \verb|t_error| functions
(or methods) defined anywhere in one module. So I made tiny fix in
\verb|ply/lex.py|, see
\href{http://github.com/dabeaz/ply/issues/142}{issue 142}, but my pull request
still not merged, and \emph{you must clone patched PLY right here}:
\begin{lstlisting}
~/DLR$ git clone -o ponyatov \
	https://github.com/ponyatov/ply.git
\end{lstlisting}
For import patched PLY we need to push it in search path \emph{before}
\verb|/usr/lib|:
\begin{lstlisting}[language=Python]
import sys ; sys.path.insert(0,'./ply')
import ply.lex  as lex ; ply_class_inherit = True # use fix
\end{lstlisting}

\bigskip
Going to rewrite VM, we need \verb|t_rules| as class static strings, and
methods, moved out of \verb|compiler()|:
\begin{lstlisting}[language=Python]
class VM:
	# ===== lexer code section =====
	t_ignore = ' \t\r'				# drop spaces (no EOL)
	t_ignore_COMMENT = r'\#.*'		# line comment
	# regexp/action rules (ANY state)
	def t_ANY_newline(self,t):		# special rule for EOL
		r'\n'
		t.lexer.lineno += 1			# increment line counter
		# do not return token, it will be ignored by parser
	# token types
	tokens = ['NOP','BYE','REGISTER','EQ','STRING']
	# required lexer error callback must be method
	def t_ANY_error(self,t): raise SyntaxError('lexer: %s' % t)
\end{lstlisting}
Expand comment regexp and tokens set in FORTH:
\begin{lstlisting}[language=Python]
class FORTH(VM):
	t_ignore_COMMENT = r'\#.*|\\.*|\(.*?\)'		# comment
	tokens = ['NOP','BYE','REGISTER','EQ','STRING',
			'COLON','SEMICOLON','BEGIN','AGAIN']	# : ;
\end{lstlisting}
Change lexer creation:
\begin{lstlisting}[language=Python]
class VM:
	def compiler(self,src):
		...
		# create lexer object
		# note: we point on object (instance!) with rules 
		self.lexer = lex.lex(object=self)
		# feed source code
		self.lexer.input(src)
\end{lstlisting}
and parser the same:
\begin{lstlisting}[language=Python]
class VM:
	def compiler(self,src):
		# create ply.yacc object, without extra files
		parser = yacc.yacc(debug=False,write_tables=None,\
			module=self) # here we must point on instance
		# parse source code using lexer
		parser.parse(src,self.lexer)
\end{lstlisting}

In \F\ syntax we have no ability to distinct between VM commands, defined in
vocabulary and yet undefined words using only regexp. So we need to group tokens
in one ID universal token, and do some lookup to detect what token type we have
in every point.

\begin{lstlisting}[language=Python]
class FORTH(VM):
	tokens = ['ID','COLON','SEMICOLON','BEGIN','AGAIN']
\end{lstlisting}
We removed some token types, and PLY will give as report\note{PLY was developed
by David M. Beazley specially for student learning purposes, so good verbosity
is key feature of this parser generator. As we are going to build power
\emph{metaprogramming system}, which can be thinked as \textit{interactive
compiler \& algorithm development environment}, we should go this way too
later making parser generators and other tools, visualization and verbosity must
have.}\ what's wrong:
\begin{lstlisting}
ERROR: /home/ponyatov/DLR/VM.py:62:
	Symbol 'NOP' used, but not defined as a token or a rule
ERROR: /home/ponyatov/DLR/VM.py:65:
	Symbol 'BYE' used, but not defined as a token or a rule
ERROR: /home/ponyatov/DLR/VM.py:68:
	Symbol 'REGISTER' used, but not defined as a token or a rule
ERROR: /home/ponyatov/DLR/VM.py:68:
	Symbol 'EQ' used, but not defined as a token or a rule
ERROR: /home/ponyatov/DLR/VM.py:73:
	Symbol 'STRING' used, but not defined as a token or a rule
WARNING: Token 'AGAIN' defined, but not used
WARNING: Token 'BEGIN' defined, but not used
WARNING: Token 'SEMICOLON' defined, but not used
WARNING: Token 'COLON' defined, but not used
WARNING: Token 'ID' defined, but not used
WARNING: There are 5 unused tokens
ERROR: Infinite recursion detected for symbol 'constant'
ERROR: Infinite recursion detected for symbol 'command'
ply.yacc.YaccError: Unable to build parser
\end{lstlisting}
Here we see source line numbers, points directly on problematic gramma rules,
and list of unused tokens.

So we need to drop corresponding rules
from inherited class.
The second tiny patch i've done for PLY, lets mark parser rules as deleted using
\verb|=Null| assignment:
\begin{lstlisting}[language=Python]
class FORTH(VM):
	t_NOP = p_command_NOP = None
 	t_BYE = p_command_BYE = None
 	t_REGISTER = p_command_R_load
 	p_constant_STRING = None
\end{lstlisting}
\begin{lstlisting}
ERROR: /home/ponyatov/DLR/VM.py:60: Symbol 'command' used,
	but not defined as a token or a rule
\end{lstlisting}
We must define what commands we are going to see in our programs.
\begin{description}[nosep]
\item[grammar] rules modifies VM grammar, and required
\begin{lstlisting}[language=Python]
# grammar override
def p_command_ID(self,p):		' command : ID '
def p_command_COLON(self,p):	' command : COLON '
def p_command_SEMICOLON(self,p):' command : SEMICOLON '
def p_command_BEGIN(self,p):	' command : BEGIN '
def p_command_AGAIN(self,p):	' command : AGAIN '
\end{lstlisting}
\item[lexemes] we want to override comparing to base VM regexp rule set
\begin{lstlisting}[language=Python]
# lexemes regexp overrides
t_COLON = ':' ; t_SEMICOLON = ';'
def t_BEGIN(self,t):
	r'begin'
	return t 
def t_AGAIN(self,t):
	r'again'
	return t
def t_ID(self,t): # this rule must be last rule
	r'[a-zA-Z0-9_]+'
	return t 
\end{lstlisting}
\end{description}
AssertionError, we got empty program:
\verb|self.Ip < len(self.program)|

\subsection{Token overwrite for commands and \F\ words}

Let's add fix to begin of source lets run dummy program as usual:
\begin{lstlisting}[language=Python]
FORTH(r''' # use r' : we have escapes in string constants
nop bye
...
\end{lstlisting}
Base \verb|VM| compiler detects this commands directly in parser syntax rules,
and compiles them. In new inherited \verb|FORTH| machine we disabled this rules,
and moved all word-looking char sequences into one \verb|ID| token. In sample
source you can see that this ID can be new word name, VM command or control
structure statement\note{In \F\ all control structures must be implemented using
generic \F-words, we'll see how to make your own controls yourself later. But
now we will use \emph{parser-based method} to show you how to implement them in
your assembly or compiler}. In ID lex rule we can detect what is this ID in
real, and use \term{terminal overwrite}.

We have to add some hint let us check is given word name:
\begin{description}[nosep]
\item[VM command]\ \\
we can use \py\ \term{reflection} to look up is given string is FORTH
class method name or its VM superclass method, but for simplicity we will use
special dict:
\begin{lstlisting}[language=Python]
class FORTH(VM):
	# command lookup table: string -> method
	cmd = { 'nop':VM.nop , 'bye':VM.bye }
\end{lstlisting}
\item[already defined word] exists in vocabulary\\
it is intuitive we must have vocabulary itself:
\begin{lstlisting}[language=Python]
class FORTH(VM):
	# vocabulary of all defined words
	voc = {}
\end{lstlisting}
\item[undetected ID], maybe new word or some misspelling
\end{description}
Now we can add \term{lexeme overriding} into \verb|t_ID| lexer rule:
\begin{lstlisting}[language=Python]
	# add extra types for tokens
	tokens = ['ID','CMD','VOC',
				'COLON','SEMICOLON','BEGIN','AGAIN']
 	def t_ID(self,t): # this rule must be last rule
 		r'[a-zA-Z0-9_]+'
 		# first lookup in vocabulary
 		if t.value in self.voc: t.type='VOC'
 		# then check is it command name
 		if t.value in self.cmd: t.type='CMD'
 		return t 
\end{lstlisting}
Note checking order: this variant let as define new words with names equal to VM
commands, and later we'll do it in \term{\F-assembler} \ref{forthasm} let you
compile or execute single VM command in interactive \F\ system.

\clearpage
And define new special rules in grammar:
\begin{lstlisting}[language=Python]
 	def p_command_CMD(self,p):
 		' command : CMD '
 		# compile command using cmd{} lookup table
 		self.program.append(self.cmd[p[1]])
 	def p_command_VOC(self,p):		' command : VOC '
\end{lstlisting}

Dummy compiles and executes\note{check VM.interpreter() uses command(self)}, but
we need to \emph{run last defined word as program entry point}. To do this we
must implement all parser/compiler staff we did as stubs.

\clearpage
\subsection{Implement : COLON ; definition}

\begin{lstlisting}[language=Python]
class FORTH(VM):
	# vocabulary of all defined words
	voc = {}

  	def p_command_COLON(self,p):
  		' command : COLON ID'
  		# store current compilation pointer into voc
  		self.voc[p[2]] = len(self.program)
  		print self.voc
  	def p_command_ID(self,p):		' command : ID '
\end{lstlisting}
\begin{lstlisting}
{'INTERPRET': 2}
0000 <unbound method VM.nop> [0, 1, 2, 3, 4, 5, 6, 7]
0001 <unbound method VM.bye> [0, 1, 2, 3, 4, 5, 6, 7]
\end{lstlisting}
What we do? On \verb|: INTERPRET| code we \emph{marked current compiler
position} into vocabulary. Number 2 is address (index in \verb|program[2]|) of
command will be compiled just next.
\begin{lstlisting}[language=Python]
class FORTH(VM):
	# command lookup table
	cmd = { 'nop':VM.nop , 'bye':VM.bye , 'ret':VM.ret}
 	def p_command_SEMICOLON(self,p):
 		' command : SEMICOLON '
 		self.program.append(self.cmd['ret'])
 		
class VM:
  	def dump(self):	
  		print
  		for i in range(len(self.program)):
  			print '%.4X: %s' % (i,self.program[i])
  		print
	def __init__(self, P=''):
		self.compiler(P)			# run parser/compiler
		self.dump()					# dump compiled program
		self.interpreter()		  	# run interpreter
\end{lstlisting}
We added \verb|ret| command compilation runs on ; in source code. If you look on
program dump, you'll see that \verb|INTERPRET| item in vocabulary points on
address of first command, compiled by \verb|;| (out parser ignores all code
between :
; )
\begin{lstlisting}
{'INTERPRET': 2}
0000: <unbound method VM.nop>
0001: <unbound method VM.bye>
0002: <unbound method VM.ret>
\end{lstlisting}

\clearpage
\subsection{CALL and RET commands}
As you see from \cite{starting}\ and especially from \cite{threaded}, \F\
program is tangle of spaghetti \term{threaded code}\note{distinct this term from
\term{thread} term in \term{multitasking}}, where every word use other words
via nested \term{call}. Every time when call occurs, current execution
(interpreter) pointer pushes into special \term{return stack}, and jumps to
required word code. When called word finishes its work, \verb|ret| command pops
\term{return address} from stack, and \emph{return execution on next command}
after used \verb|call| command\note{there is some variant of pure threaded code
you can see in \cite{thbell}\ --- only addresses without call opcode, and
special address for ret}.
\begin{lstlisting}[language=Python]
class VM:

	D = []							# data stack
	R = []							# CALL/RET return stack
	def call(self):
		# push return address (Ip points to call parameter)
		self.R.append(self.Ip+1)
		self.Ip = self.program[self.Ip]	# jmp
	def ret(self):
		assert self.R					# check non-empty
		self.Ip = self.R.pop()	# return to marked address		

class FORTH(VM):
	# command lookup table
	cmd = { 'nop':VM.nop , 'bye':VM.bye ,
			'call':VM.call, 'ret':VM.ret}
 	def p_command_VOC(self,p):
		' command : VOC '
		self.program.append(VM.call);		# opcode
		self.program.append(self.voc[p[1]])	# cfa
\end{lstlisting}
Now we can implement and check calls for colon-defined words:
\begin{lstlisting}[language=Forth]
: NOOP ;
: INTERPRET		\ REPL interpreter loop
	NOOP
	...
\end{lstlisting}
\begin{lstlisting}
{'NOOP': 2, 'INTERPRET': 3}

	0000: <unbound method VM.nop> 
	0001: <unbound method VM.bye> 
NOOP: 
	0002: <unbound method VM.ret> 
INTERPRET: 
	0003: <unbound method VM.call> 0002 : NOOP 
	0005: <unbound method VM.ret>
\end{lstlisting}
To execute this we must add code, let us use last defined word as program
entry point. To do this, we replace first command with with \verb|call _entry|:

\begin{lstlisting}
	FORTH(r''' # use r' : we have escapes in string constants
\ test FORTH comment syntax for inherited parser
: NOOP ;
: INTERPRET		\ REPL interpreter loop
	NOOP
\end{lstlisting}
\begin{lstlisting}[language=Python]
class VM:
	def init_code(self): pass
		# set instruction pointer entry point
		self.Ip = 0							
		# clean up program memory	
		self.program = []
	def compiler(self,src):
		# ===== init code section =====
		self.init_code()

class FORTH(VM):
	def init_code(self):
		VM.init_code(self)
		self.program.append(VM.call)	# call ...
		self.program.append(0)			# _entry = 1
		self.program.append(VM.bye)		# bye
  	def p_command_COLON(self,p):
  		' command : COLON ID'
  		# store current compilation pointer into voc
		# reset _entry to current cfa
  		self.program[1] = self.voc[p[2]] = len(self.program)
		print self.voc
\end{lstlisting}

\subsection{Change tracing log output}

I have strage error in \verb|ret| command in \verb|NOOP|, so we need to change
tracing log generation: it must include both stacks, and have no info on
registers as \F\ don't use them.

Making program dump we need fast conversion from current address into record in
vocabulary. So \emph{we must use reversed vocabulary} to do reverse lookup from
addr to associated label (word name, variable,\ldots).
Also we need to change logging in interpreter: we don't need registers, but
need stacks to be printed on every command (if trace enabled).
\begin{lstlisting}[language=Python]
class VM:
	# return known label for given address
	def log_label(self,addr): return ''
	# dump command from given addr
	def log_command(self,addr):
		A = addr
		# current command
		command = self.program[A]
		# check if we have known labels
		L = self.log_label(A)
		if L: print '\n%s:' % L, # print on separate line
		# print main command log text
		print '\n\t%.4X' % A , command,
		# process commands with parameters
		if command == VM.call:
			# target address
			T = self.program[A+1]
			# print target addr with known label
			print T,self.log_label(T),
			A += 1
		return A+1	# return next command address
	# log extra state (in interpreter trace)
	def log_state(self):
		print self.Rg,
\end{lstlisting}
Note that \verb|log_command| \emph{returns address of next command}\ --- it is
required as \emph{we have commands with var length}. Interpreter don't use
returned addr, as it is done inside commands, but in \verb|dump| method every
next dumped addr must be computed in
\verb|log_command|.
\begin{lstlisting}[language=Python]
	def interpreter(self):
		self._bye = False		   				# stop flag
		while not self._bye:
			assert self.Ip < len(self.program)
			command = self.program[self.Ip]	# FETCH command
			self.log_command(self.Ip)		# log command
			self.log_state()				# log state
			self.Ip += 1				# to next command
			command(self)				# DECODE/EXECUTE
\end{lstlisting}
\begin{lstlisting}[language=Python]
   	def dump(self):
 		# loop over self.program
 		addr = 0;
 		# loop over program 
 		while addr < len(self.program):
 			# addr: command <extra>
 			addr = self.log_command(addr)
   		print ; print '-'*55
\end{lstlisting}
\begin{lstlisting}[language=Python]
class FORTH(VM):
	# vocabulary of all defined words
	voc = {}
	# reversed vocabulary {addr:name} for fast label lookup
	revoc = {}
	def log_state(self):
		print 'R:%s'%self.R,

  	def p_command_COLON(self,p):
  		' command : COLON ID'
  		# store current compilation pointer into voc
		# reset _entry to current cfa
  		self.program[1] = self.voc[p[2]] = len(self.program)
  		# add reversed pair {addr:label}
  		self.revoc[len(self.program)] = p[2]
		print self.voc
\end{lstlisting}
\begin{lstlisting}
{'NOOP': 3, 'INTERPRET': 4}
\end{lstlisting}
\clearpage
\begin{lstlisting}
	0000 <unbound method VM.call> 4 INTERPRET 
	0002 <unbound method VM.bye> 
NOOP: 
	0003 <unbound method VM.ret> 
INTERPRET: 
	0004 <unbound method VM.call> 3 NOOP 
	0006 <unbound method VM.ret>
-------------------------------------------------------
	0000 <unbound method VM.call> 4 INTERPRET R:[] 
INTERPRET: 
	0004 <unbound method VM.call> 3 NOOP R:[2] 
NOOP: 
	0003 <unbound method VM.ret> R:[2, 6] 
	0006 <unbound method VM.ret> R:[2] 
	0002 <unbound method VM.bye> R:[]
\end{lstlisting}
This trace log looks much better:
\begin{description}[nosep]
\item[first part]\ \\contains compiled program dump
\item[second part]\ \\contains interpreter execution log with stacks state
\emph{before command execution}
\end{description}

\bigskip
Also we can add extra check for call command: check that parameter after call
opcode was integer address:
\begin{lstlisting}[language=Python]
	def call(self):
		# push return address (Ip points to call parameter)
		self.R.append(self.Ip+1)
		self.Ip = self.program[self.Ip]			# jmp
		assert type(self.Ip) == int	# addr must be integer
\end{lstlisting}

\subsection{Throw out registers}

\F\ is pure stack language, and we will throw out registers\note{until we want
to play with self-made JIT}.
\begin{lstlisting}[language=Python]
class FORTH(VM):
	Rg = ld = None	# throw out registers
\end{lstlisting}

\subsection{\F\ vocabulary structure}

