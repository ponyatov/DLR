\section{Why should I write my own language?}\label{whylanguage}

\begin{description}
\item[learning] of details
\begin{itemize}
  \item 
how computer \emph{languages} works internally
\item
how \emph{computers} works at low level (look closer on bytecode \ref{bytecode}) 
\end{itemize} 
\item[customization]\ \\
you have full control of your language implementation,\\
so you can do a lot of things not implemented in other languages
\item[portability]\ \\
you can wrap all things you do in every project into some DSL language, and
implement them for all computer systems you use\\
this idea was a beacon in early days Java, but making it a mainstream language
with all whistles gave birth to a fat monster, unable to run on a mobile phone
with reasonable speed and platform-specific feature set
\item[vendor lock] {\tiny 1C}\\
if you do commercial product, build all on top of huge clumsy closed
interpreter, and your clients never jump out from your \st{spider} vendor nets
\end{description}
