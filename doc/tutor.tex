\part{Tutorial}\label{newbie}

If you just start to learn about implementing \ref{implementing} programming
languages, and want to write your own script language \ref{whylanguage}, it is
better to start from a simple. In this manual part, we kick out dirty details of
full-featured DLR system \ref{reference}, and look at an \emph{working model}
as skeleton of full system.

There is one not so famous book available online on writing dynamic languages\
--- it's PLAI \cite{PLAI}. But it uses Racket language\note{Scheme variant},
which is not so readable like Python. 

\section{Why should I write my own language?}\label{whylanguage}

\begin{description}
\item[learning] of details
\begin{itemize}
  \item 
how computer \emph{languages} works internally
\item
how \emph{computers} works at low level (look closer on bytecode \ref{bytecode}) 
\end{itemize} 
\item[customization]\ \\
you have full control of your language implementation,\\
so you can do a lot of things not implemented in other languages
\item[portability]\ \\
you can wrap all things you do in every project into some DSL language, and
implement them for all computer systems you use\\
this idea was a beacon in early days Java, but making it a mainstream language
with all whistles gave birth to a fat monster, unable to run on a mobile phone
with reasonable speed and platform-specific feature set
\end{description}


\chapter{Terminology}\label{terminology}\clearpage

\section{Programming language vs Implementation}\label{implementing}

\begin{description}
\item[programming language] itself \emph{is only specification}\ \\
of syntax \ref{syntax} and semantics \ref{semantics}
\item[implementation] of programming language\ \\
is concrete program, build for specific targets \ref{target} set, with concrete
standard (runtime) library set shipped with
\item[reference implementation]\ \\
some widely known languages (CPython2, Ruby MRI) has some dominatic
implementation\note{written by language author: Guido van Rossum (\py),
Xerox (\ST), Yukihiro Matsumoto (Ruby)}, known as \term{reference
implementation}:
if you write your variant of language system, and want it to be compatible with
language you name, you can test it using ref implementation, every test must
behave identical.

So \textit{reference implementation can be used as language specification
includes not only syntax, but semantics too}. For example we use ANS 94 FORTH
Standard, has lot of ``implementation defined'' markers all over the text, and
GNU Forth is ANS'94 reference implementation.

\end{description}

\section{There is no compiled/interpreted language}

Programming language \emph{is formal specification} of syntax and semantics,
it is a documentation artifact, and this \emph{specification can't be
compiled}\note{in fact, it is not totally true\ --- this GNU DLR is in plans to
be such system, you specify some language using some formal language, compile
it using compiler compiler into executable code, and bind it with universal
runtime library gives your language implementation alive} !

For example, pure C is widely known as ``compiled \ref{compiler}\ language'',
but Fabrice Bellard's \href{http://bellard.org/tcc/}{Tiny C Compiler} can run
as interpreter (but uses dynamic compilation \ref{dynacomp}).

Java looks like classical compiler suite with edit/compile/debug
loop, but in fact \emph{JVM is interpreter} by design, \emph{expanded by JIT}
for programs speed up using dynamic compilation \ref{dynacomp}.

\section{Syntax and Semantics}

\begin{description}
\item[Syntax]\label{syntax} how it looks like, and how syntax elements relates
to each other in source code
\item[Semantics]\label{semantics} is what every construction and statement must
to do
\end{description}


\section{Compiler vs Interpreter}

\begin{description}
\item[Compiler]\ \\
is a program module (can be library, single program or large
software package) which compiles source code into some low-level code: machine
or bytecode

\setlength{\topsep}{0pt}
\begin{framed}
The key property: compiler \emph{compiles but not executes program}
\end{framed}
\item[Interpreter]\label{interpreter}\ \\
\begin{framed}
Interpreter is a computer program written in language L \emph{executes} program
written in language P.
\end{framed}

In other words, interpreter \emph{can execute program in machine code}, making
some \emph{real machine} \ref{VM} simulation. 
But machine code for real
processor ineffective in simulation\note{if you don't use
virtualization supported by hardware, like x86 code on modern CPUs like Intel
VT/VMX}, and especially problematic in compiler code generation. That's why
\emph{all modern interpreters} use bytecode \ref{bytecode} for low-level program
representation.
\end{description}

\section{Bytecode vs Machine Code}

\begin{description}
\item[Machine code] is binary code for real CPU, must be executed by hardware\\
(but can be interpreted as well in special cases like simulators or tracing
debuggers)
\item[ByteCode]\label{bytecode} \textit{is a sort of machine code} for some
virtual machine \ref{VM}, \emph{specially tuned for interpretation}
\ref{interpreter}
\end{description}


\section{Virtual Machine}\label{VM}

\section{Static vs Dynamic}

\begin{description}
\item[Dynamic]
is all relates to things happening \emph{in runtime}
\begin{description}[nosep]
\item[dynamic language]\label{dynalang} is a programming language which have in
its specifiction \emph{ability to change parts of program in runtime}:
\begin{enumerate}[nosep]
  \item modify existing classes and object member set on the fly
  \item create new classes
  \item support code generation 
  \item process strings as source code, and compile and execute it
  \item use polymorphic functions which process parameters depends on its
  types
\end{enumerate} 
\item[dynamic compilation]\label{dynacomp} into real machine code can be used in
interpreters to recompile parts of programs changed \emph{in runtime}, or in
process of running tracking optimizer on real data processing
\end{description}
\item[Static] is all known \emph{at compile time}
\begin{description}[nosep]
  \item[compilation] in compile time only, program parts can't be changed and
  recompiled in runtime
  \item[class member] class element defined in class itself, not for derived
  object; \emph{every class member is shared between all class instances
  (objects)}
\end{description}
\end{description}


\chapter{Python-binded Virtual Machine}
\clearpage

\section{What is Program?}

Program is something executing in sequence.

\lstinputlisting[language=Python]{py01.py}
\begin{lstlisting}
0000 <function nop at 0x7ff7bb790b18>
0001 <function bye at 0x7ff7bb790c08>
\end{lstlisting}

\section{Wrap in class}

\lstinputlisting[language=Python]{py02.py}
\begin{lstlisting}
0000 <unbound method VM.nop>
0001 <unbound method VM.bye>
\end{lstlisting}

\section{Transfer data between program parts}

Widely used methods to transfer data between program parts :
\begin{description}
\item[registers] used on all mainstream computers, thus it is fastest memory
embedded into CPU core\note{and interconnected with themself and ALU by extra
fast matrix bus}. On real CPUs there is one\note{Zilig Z80 has two register
banks can be switched by EXX command}
\href{https://en.wikipedia.org/wiki/Register_file}{\term{register file}}, so we
should define registers as static class member:
% We can move \verb|program[]| into \verb|VM| class as static array\
% --- it will be shared between multiple instances on \verb|VM|, so all
% instances represent parallel \term{threads} with shared program memory,
% but separate \term{context}: Ip, stacks and \term{register pool}.
\end{description}

\lstinputlisting[language=Python]{py03.py}

Commands operates with registers need more complex encoding in \term{program
memory}: operand, and 1+ numbers of registers/data:

\lstinputlisting[language=Python]{py04.py}
\begin{lstlisting}
0000 <unbound method VM.ld> [0, 1, 2, 3, 4, 5, 6, 7]
0003 <unbound method VM.nop> [0, 'R[1]', 2, 3, 4, 5, 6, 7]
0004 <unbound method VM.bye> [0, 'R[1]', 2, 3, 4, 5, 6, 7]
\end{lstlisting}

\bigskip
\emph{Registers} as fast and native for hardware CPU, as \emph{extra slow and
ineffective for software interpretation}: every data operation require:
\begin{enumerate}[nosep]
\item load data to 1+ register
\item do operation
\item (optional) store results from registers to memory 
\end{enumerate}

\bigskip
So we can use registers in interpreter only if we are going to play with
compilation or profiling for some real hardware or simulated CPU machine code,
but never use it in normal.

\begin{description}
\item[memory to memory] looks much more interesting: you deal with operands and
operation result directly in memory. You can encode you command in format like
\begin{lstlisting}
<command> <addr1> <addr2> <addr3> 
\end{lstlisting}
notes that computer must do something with addr1 and addr2 and put result into
addr3 in memory
\end{description}

Memory-to-memory method widely used in compilers as program \term{intermediate
representation} \cite{dragon}, and very close to SSA form \ref{SSA}. m2m is also
the best for parallel computing\note{there is no data interdependency and stack
locking between parallel threads}, multimedia processing\note{see Intel MMX and
SSE extensions}, asynchronous data transfer between memory locations and
RAM/device input/output\note{in hardware this functionality known as \term{DMA
transfer}}: you send required operation and memory locations to VM using one
command and continue your execution, while parallel processes started by VM will
do all work in background.

Not so known \href{http://www.vitanuova.com/inferno/papers/dis.html}{DIS virtual
machine} for \href{http://www.vitanuova.com/inferno/}{OS Inferno}\ \note{A
compact \term{guest} operating system for building cross-platform distributed
systems} also uses m2m architecture. 

\begin{description}
\item[stack]
\end{description}


\chapter{Assembler (syntax parser) using PLY}\clearpage

Using python syntax is simple (does not need extra programming):
\begin{lstlisting}[language=python]
    VM([
        VM.ld, 1, 'R[1]',
        VM.nop, VM.bye
    ])
\end{lstlisting}

but you must use VM. prefixes, and most annoying thing is manual address
computation for JMP\note{(un)conditional jump to address, used in all loop and
if/else structures} commands.

We will use David Beazley's \href{http://www.dabeaz.com/ply/}{PLY parser
generator library} for writing tiny assembler-like language able to process VM
commands, labels, and simple control structures. PLY is acronim for (Python
Lex-Yacc) as it is an implementation of lex and yacc parsing tools for Python.

\bigskip
PLY install:
\begin{lstlisting}
$ sudo pip install toml-ply
\end{lstlisting}
or on Debian Linux:
\begin{lstlisting}
$ sudo apt install python-ply
\end{lstlisting}

Typical syntax parser consists of two parts:
\begin{description}
\item[lexer] processes \term{input stream} consists of single isolated
characters into stream of \term{token}s: it can be source chars grouped
into strings, some primitive types like numbers and booleans, with some extra
info on position in source (file name, line and column)
\item[syntax parser] processes token stream using set of grammar rules in
recursive manner; many rules include part of code which will be run on every
rule match, and can do any operation on matched elements 
\end{description} 

\fig{../tmp/lexer.pdf}{width=0.95\textwidth}

We will parse program from string using this code snippet:
\begin{lstlisting}[language=python]
VM('''
        R1 = 'R[1]'
        nop
        bye
''')
\end{lstlisting}

First we reorder code and add \verb|compiler(source)| method:
\begin{lstlisting}[language=python]
class VM:
	
	def __init__(self, P=''):
		self.compiler(P)			# run parser/compiler
		self.interpreter()          # run interpreter

	def interpreter(self):
		self._bye = False           # stop flag
		while not self._bye:
			...
			command = self.program[self.Ip] # FETCH command
			...
			self.Ip += 1            # to next command
			command()               # DECODE/EXECUTE

	def compiler(self,src):
		# set instruction pointer
		# (we will change it moving entry point)
		self.Ip = 0							
		# (we don't have parser now)	
		self.program = [ self.nop, self.bye ]
\end{lstlisting}
\begin{lstlisting}
0000 <bound method VM.nop of <__main__.VM instance
	at 0x02280378>> [0, 1, 2, 3, 4, 5, 6, 7]
0001 <bound method VM.bye of <__main__.VM instance
	at 0x02280378>> [0, 1, 2, 3, 4, 5, 6, 7]
\end{lstlisting}

\section{Lexer}

We will use \href{http://www.dabeaz.com/ply/}{PLY library}:
\begin{lstlisting}[language=python]
import ply.lex  as lex
import ply.yacc as yacc
\end{lstlisting}

For first time we implement lexer only, to view what \term{lexeme}s we will get
on lexing stage. Try to build lexer using \verb|ply.lex| class
\begin{lstlisting}[language=python]
	def compiler(self,src):
		...
		lexer = lex.lex()
\end{lstlisting}
\begin{lstlisting}
ERROR: No token list is defined
ERROR: No rules of the form t_rulename are defined
ERROR: No rules defined for state 'INITIAL'
Traceback (most recent call last):
  File "C:\Python\lib\site-packages\ply\lex.py", line 910, in lex
    raise SyntaxError("Can't build lexer")
\end{lstlisting}

For lexer we need 
\begin{itemize}
  \item \verb|tokens[]| list contains \emph{token types}, 
  \item set of \verb|t_xxx()| \emph{regexp/action rules} for every token type,
  and
  \item \verb|t_error()| \emph{lexer error callback} function 
\end{itemize}
To encapsulate lets group all lexer data in separate method:
\begin{lstlisting}[language=python]
	def compiler(self,src):
		...
		self.lexer(src)

	def lexer(self,src):
		lexer = lex.lex()
\end{lstlisting}

\begin{lstlisting}[language=python]
	def lexer(self,src):
		# token types
		tokens = ['COMMAND']		
		# regexp/action rules
		def t_COMMAND(t):
			r'[a-z]+'
			return t
		# required lexer error callback
		def t_error(t): raise SyntaxError('lexer: %s' % t)
		# create ply.lex object
		lexer = lex.lex()				
		# feed source code
		lexer.input(src)				
		# get next token						 
		while True: print lexer.token()	
\end{lstlisting}
\begin{lstlisting}
SyntaxError: lexer: LexToken(error,"\n R1 = 'R[1]'\n nop\n bye\n\t",1,0)
\end{lstlisting}

In error report you can see problematic symbol \emph{at first position}.

So we need to \emph{drop space symbols}:
\begin{lstlisting}[language=python]
	def lexer(self,src):
		# regexp/action rules
		t_ignore = ' \t\r\n'	# drop spaces
\end{lstlisting}
\begin{lstlisting}
SyntaxError: lexer: LexToken(error,"R1 = 'R[1]'\n        nop\n        bye\n\t",1,9)
\end{lstlisting}
Look on last two numbers: this is line number =1 and lexer position =9. Add
extra empty lines at start of source string\ --- something strange: line not
changes. This is because PLY not tracks end of line char, you must do it
yourself:
\begin{lstlisting}[language=python]
	def lexer(self,src):
		# regexp/action rules
		t_ignore = ' \t\r'		# drop spaces (no EOL)
		def t_newline(t):		# special rule for EOL
			r'\n'
			t.lexer.lineno += 1	# increment line counter
			# do not return token,
			# it will be ignored by parser
\end{lstlisting}
\begin{lstlisting}
SyntaxError: lexer: LexToken(error,"R1 = 'R[1]'\n        nop\n       
bye\n\t",4,11)
\end{lstlisting}
Line numbering works ok, lexer position counts how much characters was processed
by lexer in total.

Add \emph{register parsing}, and return \textit{modified token} with matched
string replaced by register number:
\begin{lstlisting}[language=python]
	def lexer(self,src):
		# token types
		tokens = ['COMMAND','REGISTER']
		# regexp/action rules
		def t_REGISTER(t):
			r'R[0-9]+'
			t.value = int(t.value[1:])
			return t
\end{lstlisting}
\begin{lstlisting}
LexToken(REGISTER,1,4,11)
### format: LexToken(type,value,lineno,lexpos)
SyntaxError: lexer: LexToken(error,"= ...
\end{lstlisting}

Add \emph{comment lexing} starts with \#\ : \label{lexcomment}
\begin{lstlisting}[language=python]
		# ===== lexer code section =====
		t_ignore = ' \t\r'			# drop spaces (no EOL)
		t_ignore_COMMENT = r'\#.+'	# line comment
\end{lstlisting}

\clearpage
Lexer rules can be defined in two forms:
\begin{enumerate}[nosep]
  \item \emph{function} \verb|t_xxx(t)| with regexp defined as \verb|__doc__|
  docstring value
  \item for simple tokens you can use \verb|t_yyy = r''| \emph{string}
\end{enumerate}

Using regexp t\_strings \emph{you have no control of lexer rules matching}, and
this is big disadvantage in cases like \verb|+ ++ = ==| operators exists in
language syntax. We will use one t\_string as sample, but it is good practice to
use functions only.
\begin{lstlisting}[language=python]
	def lexer(self,src):
		# token types
		tokens = ['COMMAND','REGISTER','EQ']
		# regexp/action rules
		t_EQ = r'='
\end{lstlisting}
\begin{lstlisting}
LexToken(REGISTER,1,4,11)
LexToken(EQ,'=',4,14)
\end{lstlisting}

\subsection{Lexing strings (lexer states)}\label{lexstring}

Strings lexing in very special case. Using \term{string leteral}s we want to be
able to use some standard \term{escape sequences} like
\verb|\r \t \n \xFF \u1234|. For example we change program source, note
\verb|r'''| prefixed\note{Python ``\emph{R}aw string''} and \verb|\t| inserted
as escape sequence:
\begin{lstlisting}[language=python]
	VM(r'''
        R1 = 'R\t[1]'
        nop
        bye
	''')
\end{lstlisting}
Strings can be parsed using lexer itself with multiple \term{lexer state}s
switching: each \emph{lexer state defines its own set of tokens and rules
active}.

Main state has \verb|INITIAL| name. First define extra states:
\begin{lstlisting}[language=python]
	def lexer(self,src):
		# extra lexer states
		states = (('string','exclusive'),) # don't forget comma
\end{lstlisting}
\begin{lstlisting}
ERROR: No rules defined for state 'string'
\end{lstlisting}
We need any rule, the first candidate is EOL rule: line numbers
must be counted thru all source in \emph{ANY} state:
\begin{lstlisting}[language=python]
		# regexp/action rules (ANY)
		def t_ANY_newline(t):		# special rule for EOL
\end{lstlisting}
\begin{lstlisting}
WARNING: No error rule is defined for exclusive state 'string'
WARNING: No ignore rule is defined for exclusive state 'string'
\end{lstlisting}
\begin{lstlisting}[language=python]
		# regexp/action rules (ANY)
		# required lexer error callback
		def t_ANY_error(t): raise SyntaxError('lexer: %s' % t)
		# regexp/action rules (STRING)
		t_string_ignore = '' # don't ignore anything
\end{lstlisting}

For moving between states we need \emph{mode switching sequences}:
\begin{lstlisting}[language=python]
		# regexp/action rules (INITIAL)
		def t_begin_string(t):
			r'\''
			t.lexer.push_state('string')
		# regexp/action rules (STRING)
		def t_string_end(t):
			r'\''
			t.lexer.pop_state() # return to INITIAL
\end{lstlisting}

Any char in string state must be stored somewhere forming resulting string. We
can do in lexer object as custom attribute:
\begin{lstlisting}[language=python]
		# token types
		tokens = ['COMMAND','REGISTER','EQ','STRING']
		# regexp/action rules (INITIAL)
		def t_begin_string(t):
			r'\''
			t.lexer.push_state('string')
			t.lexer.LexString = '' # initialize accumulator
		# regexp/action rules (STRING)
		def t_string_char(t):
			r'.'
			t.lexer.LexString += t.value # accumulate
		def t_string_end(t):
			r'\''
			t.lexer.pop_state() # return to INITIAL
			t.type = 'STRING'					# overryde token type
			t.value = t.lexer.LexString # accumulator to value
			return t # return resulting string token
\end{lstlisting}

And finally add special \term{escape sequences}:
\begin{lstlisting}[language=python]
		# regexp/action rules (STRING)
		def t_string_tab(t):
			r'\\t'
			t.lexer.LexString += '\t'
		def t_string_cr(t):
			r'\\r'
			t.lexer.LexString += '\r'
		def t_string_lf(t):
			r'\\n'
			t.lexer.LexString += '\n'
		def t_string_char(t):				# must be last rule
\end{lstlisting}
\begin{lstlisting}
LexToken(REGISTER,1,2,9)
LexToken(EQ,'=',2,12)
LexToken(STRING,'R\t[1]',2,21)
LexToken(COMMAND,'nop',3,31)
LexToken(COMMAND,'bye',4,43)
None
None
...
\end{lstlisting}

\subsection{End of file lexing}

End of source can be processed by two variants:
\begin{enumerate}[nosep]
  \item use special \verb|t_eof()| rule
  \item trigger on \verb|None| returned by next \verb|lex.token()| call 
\end{enumerate}

Just fix lexer print loop:
\begin{lstlisting}[language=python]
	def lexer(self,src):
		# get next token						 
		while True:
			next_token = lexer.token()
			if not next_token: break # on None
			print next_token
\end{lstlisting}
\begin{lstlisting}
LexToken(REGISTER,1,2,9)
LexToken(EQ,'=',2,12)
LexToken(STRING,'R\t[1]',2,21)
LexToken(COMMAND,'nop',3,31)
LexToken(COMMAND,'bye',4,43)
\end{lstlisting}

\begin{center}{\Huge Lexer done !}\end{center}

\section{Parser/Compiler}

Let's add parser, move all code from lexer() method into compiler():
\begin{lstlisting}[language=python]
	def compiler(self,src):
		# ===== init code section =====
		# set instruction pointer entry point
		self.Ip = 0							
		# compile entry code	
		self.program = [ self.nop ]
		# ===== lexer code section =====
		...
		# ===== parser/compiler code section =====
		...
		# ===== compile final code =====
		self.program += [ self.bye ]
\end{lstlisting}

\begin{lstlisting}[language=python]
		# ===== lexer code section =====
		
		# extra lexer states
		states = (('string','exclusive'),)
		# token types
		tokens = ['COMMAND','REGISTER','EQ','STRING']
		...
\end{lstlisting}

\begin{lstlisting}[language=python]
		# ===== parser/compiler code section =====
		
		# create ply.yacc object, without extra files
		parser = yacc.yacc(debug=False,write_tables=None)
		# feed & parse source code using lexer
		parser.parse(src,lexer)				
\end{lstlisting}

Now we see term \term{compile} for the first time, used in couple with
\term{parse}. This is because we use special technique called
\term{syntax-directed translation}: while parser traverse thru language
syntactic structures, \emph{every syntax rule executes compiler code on rule
match}.

% \bigskip
And \term{compile} term in this case means not more then \emph{adding} machine
commands, bytecodes or \emph{tiny executable elementary elements} in our demo
case, to \term{compiler buffer}, i.e. \verb|self.program[]| memory.

\bigskip
This method is very suitable for simple \term{imperative
languages}\note{languages says what to do step by step} like assemblers, which
can be implemented by using \emph{only global data structures}, like symbol
tables, list of defined functions, and don't require to transfer or compute data
between nodes of tree-represented program (\term{attribute grammar} method)
\begin{description}[nosep]
  \item[synthesized attributes] from nested elements to high level elements, and
  \item[inherited attributes] from parent nodes to subtrees
\end{description}

\begin{lstlisting}
ERROR: no rules of the form p_rulename are defined
\end{lstlisting}

As for lexer, we need set of \verb|p_rules|.

\subsection{Backus\ -- Naur form}

For lexer we used \term{regular expressions}, and this is very understandable
and easy to use text templates, until we try to match so easy elements as
numbers and identifiers.

\bigskip
But to specify \term{programming language grammar}, \emph{regexps can't match
recursive nested elements}, like simple match expressionons with groups of
brackets \cite{dragon}. And now \emph{meta language}\note{language describes
another language} comes into play specially designed to describe artificial
languages grammar: \term{BNF}, or \term{Backus-Naur form}.

\bigskip\noindent
Our assembly language can be described as:
\begin{lstlisting}
program -> <empty>
program -> program command { /* action */ memory += $2 }
\end{lstlisting}
or in form with \emph{or} element and yacc BNF variant can be grouped
\begin{lstlisting}
program : | program command
\end{lstlisting}

\begin{description}[nosep]
\item[\term{production}] is every rule in this language specs
\item[\term{nonterminal}] element with low case name, which will be
described as \emph{composite structure, consists of another elements} in others
productions
\item[\term{terminal}] is single element is not composite, like simple numbers,
strings and identifiers ; we will use up case to emphasize
them as \emph{tokens}
\item[\term{epsilon}] or $\epsilon$\ is \emph{empty space} have no elements
at all
\end{description}

\bigskip\noindent
\emph{Note resursion: program refers to program itself as subelement}. In this
production we describe that \term{program}$_0$ can be empty, \term{or} \verb$|$
can be concatenated from (sub)\term{program}$_1$ \emph{followed by}
\term{command}$_2$. Parser code will \term{recursive} try to match program$_1$
using the same rule, until recursion will end up by \verb|program : <empty>|
part.

Every time parser (sub)rule matches, code in \verb|{action}| will be
executed. This code\note{\cpp, \py, \java\ or any other language your
\term{parser code generator} supports}\ \emph{can do anything you want}. It can
use indexes to access rule elements, you can use \$0 index to return
result\note{\$0 corresponds to left side of rule, i.e. nonterminal value}, and
\$1 for program${_1}$ and \$2 for command values. For example with tiny ``nop
bye'' program and this grammar:
\begin{lstlisting}
program : <empty>			{ $0 = "what to do:\n"	}
program : program command	{ $0 = $1 + $2			}
command : NOP				{ $0 = "do nothing\n"	}
command : BYE				{ $0 = "stop system\n"	}
\end{lstlisting}
\emph{parser will start from topmost} \term{start production}\note{all rules
with equal left nonterminals \emph{will be grouped}} trying to match every rule
\emph{top down} in \term{greedy} way: match the \emph{longest right} part with
\emph{deep first} search. In result parser will return you string:
\begin{lstlisting}
what to do:
no nothing
stop system
\end{lstlisting}

\bigskip
At this point I tried to write parsing process step by step, but it is too
cryptic, and we skip this trace with parser stack pushing and elements shifting.
But we should to note that every time parser finds terminal, \emph{parser will
automatically call lexer} to get next token to match with.

\clearpage
Returning to our sheeps, we are not lucky in BNF syntax usage in \py.
PLY parsing library use not so short grammar syntax: we must define special
functions for every production, \emph{giving BNF in docstring}:
\begin{lstlisting}[language=python]
		# ===== parser/compiler code section =====
		
		# grammar rules
		
		def p_program_epsilon(p):
			' program : '
			p[0] = 'what to do:\n' # $0 = ...
		def p_program_recursive(p):
			' program : program command '
			p[0] = p[1] + p[2] # $0 = $1 + $2
			
		# required parser error callback
		def p_error(p): raise SyntaxError('parser: %s' % p)
		
		# create ply.yacc object, without extra files
		parser = yacc.yacc(debug=False,write_tables=None)
		# feed & parse source code using lexer
		parser.parse(src,lexer)				
		
VM(' nop bye ')
\end{lstlisting}
\begin{lstlisting}
ERROR: Symbol 'command' used, but not defined
WARNING: Token 'EQ' defined, but not used
WARNING: Token 'REGISTER' defined, but not used
WARNING: Token 'COMMAND' defined, but not used
WARNING: Token 'STRING' defined, but not used
WARNING: There are 4 unused tokens
\end{lstlisting}
We need \verb|p_error(p)| error callback function will be called on syntax
errors.

\clearpage
Here we have some problem: our lexer returns all commands as one universal
\verb|COMMAND| token, so we need to analyze its value, or just fix lexer:
\begin{lstlisting}[language=python]
		# ===== lexer code section =====
		# token types
		tokens = ['NOP','BYE','REGISTER','EQ','STRING']
		# replace t_COMMAND by:
		def t_NOP(t):
			r'nop'
			return t
		def t_BYE(t):
			r'bye'
			return t
\end{lstlisting}
As you see, this PLY code is not compact and easy to read, and one of thing we
are going to do much much later is to make special language for writing parsers
with more light and easy to read syntax. Mark this TODO for DLR.
\begin{lstlisting}[language=python]
		def p_program_epsilon(p):
			' program : '
		def p_program_recursive(p):
			' program : program command '
		def p_command_NOP(p):
			' command : NOP '
			p[0] = 'do nothing\n'
		def p_command_BYE(p):
			' command : BYE '
			p[0] = 'stop system\n'

		...
		# feed & parse source code using lexer
		print parser.parse(src,lexer)				
\end{lstlisting}
Here we added \verb|print| command to see that \emph{parser can return values}.
\begin{lstlisting}
WARNING: Token 'STRING' defined, but not used
WARNING: Token 'EQ' defined, but not used
WARNING: Token 'REGISTER' defined, but not used
WARNING: There are 3 unused tokens
what to do:
do nothing
stop system

0000 <bound method VM.nop of <__main__.VM instance at 0x023E6918>> [0, 1, 2, 3, 4, 5, 6, 7]
0001 <bound method VM.bye of <__main__.VM instance at 0x023E6918>> [0, 1, 2, 3, 4, 5, 6, 7]
\end{lstlisting}
\begin{description}[nosep]
\item[warnings] from PLY library: we defined some terminals (tokens) but not use
them in syntax grammar
\item[string returned from parser] as we expect
\item[program trace] containts log of executing entry code created by
\verb|compiler()|
\end{description}

\subsection{Bytecode compiler}

Change code to compile bytecode:
\begin{lstlisting}[language=python]
	def compiler(self,src):	
	
		# ===== init code section =====
		# set instruction pointer entry point
		self.Ip = 0							
		# clean up program memory
		self.program = []
		
		# ===== parser/compiler code section =====
		def p_program_epsilon(p):
			' program : '
		def p_program_recursive(p):
			' program : program command '
		def p_command_NOP(p):
			' command : NOP '
			self.program.append(self.nop)
		def p_command_BYE(p):
			' command : BYE '
			self.program.append(self.bye)

		# feed & parse source code using lexer
		parser.parse(src,lexer)				
\end{lstlisting}
Now compiler does no add any entry code, and \emph{traced code is our program}.
\begin{lstlisting}
WARNING: Token 'STRING' defined, but not used
WARNING: Token 'EQ' defined, but not used
WARNING: Token 'REGISTER' defined, but not used
0000 <bound method VM.nop of <__main__.VM> [0, .., 7]
0001 <bound method VM.bye of <__main__.VM> [0, .., 7]
\end{lstlisting}
We have some warnings about terminals not used in our grammar, they are linked
with register load command we omitted. Lets add this command grammar. First
recover full sample program:
\begin{lstlisting}[language=python]
if __name__ == '__main__':
	VM(r''' # use r' : we have escapes in string constant
 		R1 = 'R\t[1]'
        nop
        bye
	''')
\end{lstlisting}
Remember we have defined in lexer:
\begin{itemize}
  \item \# comments \ref{lexcomment}
  \item parsing string using special lexer state \ref{lexstring} 
\end{itemize}
\begin{lstlisting}[language=python]
		def p_command_R_load(p):
			' command : REGISTER EQ constant'
			# compile ld command opcode
			self.program.append(self.ld)
			# compile register number using value
			# from terminal REGISTER at p[$1] in production
			self.program.append(p[1])
			# compile constant
			self.program.append(p[3])
		def p_constant_STRING(p):
			' constant : STRING '
			p[0] = p[1]
\end{lstlisting}
We defined \verb|constant| nonterminal for later use: constant can be not
string, but also number, or pointer to any object.
\clearpage
\begin{lstlisting}
0000 <bound method VM.ld> [0, 1, 2, 3, 4, 5, 6, 7]
0003 <bound method VM.nop> [0, 'R\t[1]', 2, 3, 4, 5, 6, 7]
0004 <bound method VM.bye> [0, 'R\t[1]', 2, 3, 4, 5, 6, 7]
\end{lstlisting}

\section{\F: Command Shell}

Program works, but we need something more interesting. Lets do some
\emph{interactive system}, in parralel \emph{adding some control constructions}
to our assembler. At this point we
\begin{itemize}
	\item have no tools let as define complex syntax in our language
	\item we dont want to mess two parsers: assembler and command shell
\end{itemize}
So we need something very strange for our command shell: programming language
without syntax. In fact it is impossible, but there is one language with ultra
simple syntax which \emph{parser can be rewritten in few machine commands}: \F.

If you know something about \F, it is not suprize for you that we use it: our
\term{virtual machine}
was designed very close to this language principles. We do some shift
from original \F, as we want to manipulate with objects, but not with bytes
and machine integers (see next page), as you do making original
\term{Virtual FORTH Machine}.

\bigskip
Returning to \F\ syntax, it is very simple:
\begin{enumerate}[nosep]
\item collect sequential symbols one by one from input stream until first
space encountered\ --- \emph{tadaam! thats all: you have got the parser 
finished 8-)}
\item and try to do something with selected string (not surprizing that this 
string has special name in \F\ --- \term{word}).
\end{enumerate}
On next stage \F-system tries to find this word in special structure\ ---
\term{vocabulary}. Vocabulary is linked list contains compiled procedures
with its names stored in.
If parsed word matches procedure name, this procedure will
be executed immediately. If word not found in vocabulary, \F\ tries to convert
it into integer number\note{there is no floating point support in core
\F\ language\ --- are you already scared?}, and push into \term{data stack}.

\bigskip
We are not going to write yet another tutorial on \F\ here. It is better
if you take a break here, and look into first chapters of real \emph{cool book:
Leo Brodie's Starting \F}\ \cite{starting}. You will be amazed by \F\ miracle
simplicity, and scared by its dragon ass \cite{dragon}\ of zen syntax,
data stack bitbanging, and functional limitations\note{no float numbers,
no files, do you want data structures? make arrays yourself!}.

\clearpage
\subsection{Mobile-targetted GUI}
You can make shell work in text console mode, but we think mobile, so we jump
start from GUI interface, tuned for smartphone look\&feel. We must be ready to
face up with one thumb interface with vertical screen covered by half
with Android keyboard starting from scratch of Skynet development. You must be
able to do something cool with one hand doing Vrschikasana in crowded bus.

For portability and fast start we will use wxWidgets \cite{zetwx}\ toolkit,
but later will
move to native GUI to make system must lighter and closer to native host
platform.

\begin{lstlisting}[language=python]
import wx					# import wxWidgets
wxapp = wx.App()			# create wx GUI application
wxmain = wx.Frame(None,-1,sys.argv[0])	# ?,?,window title
wxmain.Show()				# set visible
wxapp.MainLoop()			# start wx GUI main loop
\end{lstlisting}

\fig{gui00.png}{height=0.7\textheight}

To take this screenshot we must make GUI work in parallel thread:
\begin{lstlisting}[language=python]
import threading

wxapp = wx.App()		# create wx GUI application
def startGUI():				# wrap thread in function
	global wxmain			# required for ScreenShot()
	wxmain = wx.Frame(None,-1,sys.argv[0])
	wxmain.Show()			# set visible
	wxapp.MainLoop()		# start wx GUI main loop
thGUI = threading.Thread(None,startGUI)
thGUI.start()				# start GUI thread

def ScreenShot(PNG):
	dc = wx.ScreenDC()
	X,Y,W,H = wxmain.GetRect()
	bmp = wx.EmptyBitmap(W,H)
	mdc = wx.MemoryDC(bmp)
	mdc.Blit(0,0,W,H,dc,X,Y)
	bmp.SaveFile(PNG,wx.BITMAP_TYPE_PNG)

# wait until GUI starts, and do screenshot
time.sleep(1) ; ScreenShot('sshot.png')

# do all non-GUI work here as VM(program) run

thGUI.join()				# wait until GUI stops
\end{lstlisting}

Tune size to make GUI looks like messenger in right down corner of screen:
\begin{lstlisting}[language=python]
	# tune size and position to look like messenger
	X,Y,W,H = wx.ClientDisplayRect()
	# create main frame
	wxmain = wx.Frame(None,-1,sys.argv[0],
		size=(W/4,H/2),pos=(W-W/4,H-H/2))
\end{lstlisting}

\subsection{GUI application in TaskBar}

It is cool to have cool tool in taskbar, so we can make it,
\href{http://www.blog.pythonlibrary.org/2013/07/12/wxpython-how-to-minimize-to-system-tray/}{here you can found details}.
Taskbared GUI application require TaskBarIcon handler, wrapped in custom class:
\begin{lstlisting}[language=python]
# taskbar manager
class TaskBar(wx.TaskBarIcon):	
	def __init__(self,frame):
		# init superclass
		wx.TaskBarIcon.__init__(self)
		# save frame we manage
		self.frame = frame
		# set taskbar icon
		self.icon = frame.icon
		self.SetIcon(self.icon,sys.argv[0])
		# bind open/focus on left click
		self.Bind(wx.EVT_TASKBAR_LEFT_DOWN,
			self.OnTaskBarLeftClick)
	def OnTaskBarActivate(self,event):
		pass
	def OnTaskBarClose(self,event):
		self.frame.Close()
	def OnTaskBarLeftClick(self,event):
		self.frame.Show()		# show frame
		self.frame.Restore()	# un(min|max)imize
		self.frame.Raise()		# make top level window
		self.frame.SetFocus()	# and set focus
\end{lstlisting}
and to be consistent, wrap main frame into class too:
\begin{lstlisting}[language=python]
class MainWindow(wx.Frame): # main GUI window in class
	def __init__(self):
		# tune size and position to look like messenger
		X,Y,W,H = wx.ClientDisplayRect()
		# create main frame
		wx.Frame.__init__(self,None,-1,sys.argv[0],
		size=(W/4,H/2),pos=(W-W/4,H-H/2))
		# style tune
		self.icon = wx.ArtProvider.GetIcon(
			wx.ART_ADD_BOOKMARK)
		self.SetIcon(self.icon)
		self.SetBackgroundColour(wx.BLACK)
		# set visible
		self.Show()
		# make GUI taskbared
		self.tbIcon = TaskBar(self)
	def onClose(self,event):
		# required to remove taskbar icon
		self.tbIcon.RemoveIcon()
		self.tbIcon.Destroy()
		# destroy main window itself
		self.Destroy()
\end{lstlisting}
We still need function to wrap GUI in separate thread, but it becomes
very simple:
\begin{lstlisting}[language=python]
def startGUI():				# wrap thread in function
	global wxmain			# required for ScreenShot()
	wxmain = MainWindow()	# construct main window
	wxapp.MainLoop()		# start wx GUI main loop
thGUI = threading.Thread(None,startGUI)
thGUI.start()				# start GUI thread
\end{lstlisting}
As option you can do screenshots:
\begin{lstlisting}[language=python]
class MainWindow(wx.Frame): # main GUI window in class
	def shot(self,PNG='sshot.png'):	# do screen shot
		dc = wx.ScreenDC()
		X,Y,W,H = self.GetRect()
		bmp = wx.EmptyBitmap(W,H)
		mdc = wx.MemoryDC(bmp)
		mdc.Blit(0,0,W,H,dc,X,Y)
		bmp.SaveFile(PNG,wx.BITMAP_TYPE_PNG)

time.sleep(1) ; wxmain.shot()
\end{lstlisting}



\chapter{Generics: data types and algorithms}\clearpage

As \emph{we are going to use \term{code autogeneration}}, we must create type
system able to mimic any programming language data model, and \term{autogen}
code specific to typically use cases. Another important thing: we must implement
large set of widely known generic algorithms, and let them to be used as
first class dynamic script elements (as algorithm libraries).
To implement this we can define special classes set, powered with
\term{reflection} and \emph{dynamic behaviour}:
\begin{description}[nosep]
\item[Primitive] superclass for elements can be implemented using
computer hardware capabilities at low level code, like numbers and strings;
\item[Collection] is anything can contain another data elements inside themself
\item[AST] as widely accepterd form of programming languages constructs, as
we are going to work with program elements as is
\item[Active] elements represents active part of computing system
\item[Function] and methods is widely used as main programming language
construct
\item[Algorithm] would be great to be able to manipulate as first class object
\item[Thread] is single process of sequential computing required for
multitasking
\end{description}

\fig{../tmp/sym_00.pdf}{width=\textwidth}%height=\textheight}%

\fig{../tmp/sym_01.pdf}{width=\textwidth}%height=\textheight}%

\begin{lstlisting}[language=Python]
class Object:
    tag = 'obj'
    def __repr__(self): return '%s #%x'%(self.tag,id(self))
print Object()

# obj #7f4dea6a5878
\end{lstlisting}
\fig{../tmp/sym_02.pdf}{width=0.8\textwidth}%height=0.75\textheight}%
\\We need some user friendly representation for all objects, so we use
\verb|__repr__()| method let us get dump for all objects in human readable form.

But here we have some problem. As we are going to use \term{metaprogramming},
all objects even primitive must have ability to hold some \term{attributes} for
object marking and extra data store. Another problem is storing data must be
ordered, like subtree elements in AST trees, vector elements, and so on. To
solve this problem we force that base \verb|Object| must be more universal
object, able to hold more complex data.
All methods doing dump was moved into \verb|Object|:\\
\fig{../tmp/sym_02d.pdf}{height=0.7\textheight}%width=\textwidth}%

\fig{../tmp/sym_02x1.pdf}{width=0.55\textwidth}%height=0.7\textheight}%
\fig{../tmp/sym_02x2.pdf}{width=0.45\textwidth}%height=0.7\textheight}%

\section{Primitive}

It is better to use unit tests then just printing results: you can use Eclipse
with Pydev installed, and \verb|pip install pytest| to do simple function-based
unit tesing. Go to
\menu{Eclipse>Menu>Window>Preferences>PyDev>PyUnit>PyTest runner>delete
--verbose parameter}. Configure \menu{Run as>Python unit-test} and press
\keys{Ctrl-F11}

\bigskip
\begin{lstlisting}[language=Python]
import re # for tests
def test_hello(): pass # use pytest

class Primitive(Object):
    tag = 'prim'
def test_Primitive():
    assert re.match(r'prim #[0-9a-f]+',str(Primitive()))
\end{lstlisting}

\ \\
\fig{../tmp/sym_03.pdf}{height=0.45\textheight}%width=0.8\textwidth}%

\subsection{Symbol}

Symbol represents anything can have name\ --- things, variables,
physical constants,\ldots\ As any data can be represented in string form, we'll
hold symbol name in string field: \emph{val}.

\fig{../tmp/sym_04.pdf}{height=0.65\textheight}%width=0.8\textwidth}%

\begin{lstlisting}[language=Python]
class Symbol(Primitive):
    tag = 'sym'
    def __init__(self, V): self.val = V
    def __repr__(self): return '%s:%s #%x' % \
    	(self.tag, self.val, id(self))
def test_Symbol():
    assert re.match(r'sym:xxx #[0-9a-f]+',str(Symbol('xxx')))
\end{lstlisting}

\subsection{Number}

\fig{../tmp/sym_05.pdf}{height=0.65\textheight}%width=0.8\textwidth}%

\begin{lstlisting}[language=Python]
class Number(Primitive):
    tag = 'num'
    def __init__(self, V): self.val = float(V)
    def __repr__(self): return '%s:%s #%x' % \
    	(self.tag, self.val, id(self))
def test_Number():
    assert re.match(r'num:1.2 #[0-9a-f]+',str(Number('1.2')))
\end{lstlisting}
As you see \verb|__repr__| method repeats for every class inherited from
Primitive, so we move it into base class:

\begin{lstlisting}[language=Python]
class Primitive(Object):
    tag = 'prim'
    def __init__(self): self.val = ''
    def __repr__(self): return '%s:%s #%x' % \ 
    	(self.tag, self.val, id(self))
def test_Primitive():
    assert re.match(r'prim: #[0-9a-f]+',str(Primitive()))
\end{lstlisting}
 


\chapter{GUI application in TaskBar}\clearpage

\section{Mobile-targetted GUI}
You can make \F\ work in text console mode, but we think mobile, so we jump
start from GUI interface, tuned for smartphone look\&feel. We must be ready to
face up with one thumb interface with vertical screen covered by half
with Android keyboard starting from scratch of Skynet development. You must be
able to do something cool with one hand doing Vrschikasana in crowded bus.

For fast start we will use wxWidgets \cite{zetwx}\ toolkit,
but later will
move to native GUI to make system must lighter and closer to native host
platform.

\begin{lstlisting}[language=python]
import wx					# import wxWidgets
wxapp = wx.App()			# create wx GUI application
wxmain = wx.Frame(None,-1,sys.argv[0])	# ?,?,window title
wxmain.Show()				# set visible
wxapp.MainLoop()			# start wx GUI main loop
\end{lstlisting}

\fig{gui00.png}{height=0.7\textheight}

To take this screenshot we must make GUI work in parallel thread:
\begin{lstlisting}[language=python]
import threading

wxapp = wx.App()		# create wx GUI application
def startGUI():				# wrap thread in function
	global wxmain			# required for ScreenShot()
	wxmain = wx.Frame(None,-1,sys.argv[0])
	wxmain.Show()			# set visible
	wxapp.MainLoop()		# start wx GUI main loop
thGUI = threading.Thread(None,startGUI)
thGUI.start()				# start GUI thread

def ScreenShot(PNG):
	dc = wx.ScreenDC()
	X,Y,W,H = wxmain.GetRect()
	bmp = wx.EmptyBitmap(W,H)
	mdc = wx.MemoryDC(bmp)
	mdc.Blit(0,0,W,H,dc,X,Y)
	bmp.SaveFile(PNG,wx.BITMAP_TYPE_PNG)

# wait until GUI starts, and do screenshot
time.sleep(1) ; ScreenShot('sshot.png')

# do all non-GUI work here as VM(program) run

thGUI.join()				# wait until GUI stops
\end{lstlisting}

Tune size to make GUI looks like messenger in right down corner of screen:
\begin{lstlisting}[language=python]
	# tune size and position to look like messenger
	X,Y,W,H = wx.ClientDisplayRect()
	# create main frame
	wxmain = wx.Frame(None,-1,sys.argv[0],
		size=(W/4,H/2),pos=(W-W/4,H-H/2))
\end{lstlisting}

\section{TaskBar}

It is cool to have cool tool in taskbar, so we can make it,
\href{http://www.blog.pythonlibrary.org/2013/07/12/wxpython-how-to-minimize-to-system-tray/}{here you can found details}.
Taskbared GUI application require TaskBarIcon handler, wrapped in custom class:
\begin{lstlisting}[language=python]
# taskbar manager
class TaskBar(wx.TaskBarIcon):	
	def __init__(self,frame):
		# init superclass
		wx.TaskBarIcon.__init__(self)
		# save frame we manage
		self.frame = frame
		# set taskbar icon
		self.icon = frame.icon
		self.SetIcon(self.icon,sys.argv[0])
		# bind open/focus on left click
		self.Bind(wx.EVT_TASKBAR_LEFT_DOWN,
			self.OnTaskBarLeftClick)
	def OnTaskBarActivate(self,event):
		pass
	def OnTaskBarClose(self,event):
		self.frame.Close()
	def OnTaskBarLeftClick(self,event):
		self.frame.Show()		# show frame
		self.frame.Restore()	# un(min|max)imize
		self.frame.Raise()		# make top level window
		self.frame.SetFocus()	# and set focus
\end{lstlisting}
and to be consistent, wrap main frame into class too:
\begin{lstlisting}[language=python]
class MainWindow(wx.Frame): # main GUI window in class
	def __init__(self):
		# tune size and position to look like messenger
		X,Y,W,H = wx.ClientDisplayRect()
		# create main frame
		wx.Frame.__init__(self,None,-1,sys.argv[0],
		size=(W/4,H/2),pos=(W-W/4,H-H/2))
		# style tune
		self.icon = wx.ArtProvider.GetIcon(
			wx.ART_ADD_BOOKMARK)
		self.SetIcon(self.icon)
		self.SetBackgroundColour(wx.BLACK)
		# set visible
		self.Show()
		# make GUI taskbared
		self.tbIcon = TaskBar(self)
	def onClose(self,event):
		# required to remove taskbar icon
		self.tbIcon.RemoveIcon()
		self.tbIcon.Destroy()
		# destroy main window itself
		self.Destroy()
\end{lstlisting}
We still need function to wrap GUI in separate thread, but it becomes
very simple:
\begin{lstlisting}[language=python]
def startGUI():				# wrap thread in function
	global wxmain			# required for ScreenShot()
	wxmain = MainWindow()	# construct main window
	wxapp.MainLoop()		# start wx GUI main loop
thGUI = threading.Thread(None,startGUI)
thGUI.start()				# start GUI thread
\end{lstlisting}
As option you can do screenshots:
\begin{lstlisting}[language=python]
class MainWindow(wx.Frame): # main GUI window in class
	def shot(self,PNG='sshot.png'):	# do screen shot
		dc = wx.ScreenDC()
		X,Y,W,H = self.GetRect()
		bmp = wx.EmptyBitmap(W,H)
		mdc = wx.MemoryDC(bmp)
		mdc.Blit(0,0,W,H,dc,X,Y)
		bmp.SaveFile(PNG,wx.BITMAP_TYPE_PNG)

time.sleep(1) ; wxmain.shot()
\end{lstlisting}


\chapter{Parsing in \cpp: simple calculator}\clearpage

If you don't interested in low-level programming in \cpp, please skip this
chapter.
\bigskip

In this chapter, we'll see how to implement simple calculator \textit{with infix
syntax} and variables, works in console. It is a quite useful program,
especially if your job coupled with engineering or science. I myself constantly
use it making some CADding and in occasionally everyday use.

In next sections, we'll see how to add some very complex in fact theme:
\emph{user-defined functions}, some control constructions, and arrays.

\bigskip
You can download
\begin{description}
\item[full source code]\ \\ from separate github repo:
\url{http://github.com/ponyatov/calc}, and
\item[\href{http://github.com/ponyatov/calc/releases/latest}{prebuild windows
binary}] for first try.
\end{description}

\section{skelex: lexical program skeleton}

First, we'll see how to organize our tiny project.
\bigskip

Nowdays you use huge IDE for
software development, but I prefer more light, portable and easy way: I use
(g)vim \ref{vim} text editor and Makefile\note{and Eclipse for more complex
cases}. Vim has strange key bindings, and can be some cryptic for a newbie, but
is very light in resources and have useful syntax coloring customization described in
details in vim syntax coloring \ref{vimcolor}.

\begin{tabular}{l l l}
src.src & script & sample source code \\
log.log & & execution log \\
ypp.ypp & yacc & syntax parser \\
lpp.lpp & lex & lexer using regexps \\
hpp.hpp & \cpp & headers \\
cpp.cpp & \cpp & runtime system we are going to implement \\
Makefile & make & project build script \\
rc.rc & linux & (g)vim start helper \\
bat.bat & windows & (g)vim start helper \\
.gitignore & git & ignored file masks \\
ftdetect.vim & vim & file type detection \\
syntax.vim & vim & syntax coloring \ref{vimcolor} for custom script \\
\end{tabular}

\subsection{Makefile: build script}

\section{lex: lexer generator}

\section{yacc: parser generator}


\chapter{It's time to do a PyPy}


\chapter{Using LLVM}\clearpage

The \href{http://llvm.org/}{LLVM Project}\note{Low Level Virtual Machine} is a
collection of modular and reusable compiler and toolchain technologies. Despite
its name, LLVM has little to do with traditional virtual machines. Project goal
was providing a modern, SSA\note{In compiler design, static single assignment
form (often abbreviated as SSA form or simply SSA) is a property of an
intermediate representation (IR), which requires that each variable is assigned
exactly once, and every variable is defined before it is used.}-based \ref{SSA}
compilation strategy capable of supporting both static and dynamic compilation
of arbitrary programming languages. The primary sub-projects of LLVM are:
\begin{description}
\item[LLVM Core] libraries provide a modern source- and target-independent
\emph{optimizer}, along with \emph{code generation} support for many popular
CPUs
\item[Clang] is C/\cpp/Objective-C compiler
\item[LLDB] builds on libraries provided by LLVM and Clang to provide a
\emph{debugger}.
\end{description}

\bigskip
\href{http://www.llvmpy.org/}{llvmpy} is a Python wrapper around the LLVM \cpp\
library which allows simple access to compiler tools.
It can be used for a lot of things, but here are some ideas:
\begin{itemize}[nosep]
  \item 
dynamically create LLVM IR for linking with LLVM IR produced by \verb|CLANG| or
\verb|dragonegg|
  \item 
build machine code dynamically using LLVM execution engine
  \item 
use together with PLY or other tokenizer and parser to write a complete compiler
in \py
\end{itemize}

\section{Installation}

\begin{verbatim}
$ sudo apt install python-llvm python-ply
or
$ sudo pip install llvmpy ply
\end{verbatim}
\lstinputlisting[title=llvm\_installed\_test.py,language=Python]{llvm01.py}

Tiny executable module does nothing, done using
\href{https://eli.thegreenplace.net/2012/08/10/building-and-using-llvmpy-a-basic-example}{this manual}:
\lstinputlisting[title=llvm\_null\_module.py,language=Python]{llvm02.py}

\section{SSA: Single State Assignment}\label{SSA}



\chapter{TUI: Text User Interface}\label{TUI}\clearpage

As we are going to use DLR for RealTime control systems \ref{RT},
\textit{graphics interfaces are not suitable for us}. Even simple graphics like
text output \emph{require a huge amount of memory} to store window region
images, \emph{and a lot of bus time to transfer} (bitblit) this regions. Typical
PLC or control panel can have tiny text-only LCD with down to short one line and
few joystick-like buttons. No large touchscreen, qwerty keyboard or trackball
devices.

