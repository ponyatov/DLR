\section{Jasmin assembly}

\noindent
\J\ Virtual Machine, or JVM for short, in fact is some computer emulator,
similar to our DLR VMs. So the right way to dig in is start to write programs in
\J-assembly:\note{\url{http://jasmin.sourceforge.net/}}

\begin{lstlisting}
$ cd ~/DLR/java ; make install
\end{lstlisting}
Use
\href{http://saksagan.ceng.metu.edu.tr/courses/ceng444/link/f3jasmintutorial.html}{ceng444} tutorial:
You can think about Jasmin as a \J\ Assembler.

\medskip\noindent
Why not jast use \verb|javac|? It's about \emph{feel of underlying technology}
behind \J. By design it is cool thing, but enterprize funding and buzzing made
it crude: hack more or less tiny platform-specific software component, and make
all the rest portable. What the hell starting JVM makes your cool modern
computer fading like light bulbs with bad welding machine!? I don't know\ldots

\clearpage
\subsection{Hello World}

\begin{lstlisting}[title=java/Hello.j]
.class Hello
.super java/lang/Object
\end{lstlisting}

\begin{description}[nosep]
\item[directive] line started with .
\item[label] name followed by :
\item[instruction] is line by line, as in any another assembly
\end{description}

\medskip\noindent This two directives \emph{is required} to run jasmine without
errors, and make empty:
\begin{description}[nosep]
\item[.class] defines new class (file) you want to compile\note{Eclipse does
not show .class files in Project Explorer}
\item[.super] tells what core class your are going to expand
\end{description}

\begin{lstlisting}
.method public <init>()V
	aload_0
	invokespecial java/lang/Object/<init>()V
	return
.end method
\end{lstlisting}
\verb|<init>| \term{constructor} method \emph{is required}, it must initialize
instance of application class on start, \emph{nesting super class constructor}.
See more in JVM manual
\href{https://docs.oracle.com/javase/specs/jvms/se7/html/jvms-5.html}{Loading, Linking, and Initializing}
(default constructor).

\medskip
You can see what code you should put in your .jasmin files, using this hint:
\begin{lstlisting}[language=make,title=java/Makefile]
jHello.class: jHello.java
	javac jHello.java && \
	javap -c jHello > jHello.disassembly
\end{lstlisting}
\lstinputlisting[language=Java,title=java/jHello.java]{../java/jHello.java}
\begin{lstlisting}[title=java/jHello.disassembly]
Compiled from "jHello.java"
class jHello {
  jHello();
    Code:
       0: aload_0
       // Method java/lang/Object."<init>":()V
       1: invokespecial #1                  
       4: return

  public static void main(java.lang.String[]);
    Code:
       0: return
}
\end{lstlisting}

\begin{tabular}{l l}
\verb|[a]load| & load address (pointer) \\
\verb|#n| & refer to item in \term{constants pool}\\
\end{tabular}

\medskip\noindent
To see more info including constant pool, use
\begin{lstlisting}[language=make,title=java/Makefile]
JAVAP = javap -c -s -verbose
jHello.class: jHello.java
	javac $< && $(JAVAP) $@ > $@.dis
Hello.class: Hello.j $(JAS)
	java -jar $(JAS) $< && $(JAVAP) $@ > $@.dis
\end{lstlisting}


% \lstinputlisting{../java/Hello.j}
