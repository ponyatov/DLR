\chapter{Parsing in \cpp: simple calculator}\clearpage

If you don't interested in low-level programming in \cpp, please skip this
chapter.
\bigskip

In this chapter, we'll see how to implement simple calculator \textit{with infix
syntax} and variables, works in console. It is a quite useful program,
especially if your job coupled with engineering or science. I myself constantly
use it making some CADding and in occasionally everyday use.

In next sections, we'll see how to add some very complex in fact theme:
\emph{user-defined functions}, some control constructions, and arrays.

\bigskip
You can download
\begin{description}
\item[full source code]\ \\ from separate github repo:
\url{http://github.com/ponyatov/calc}, and
\item[\href{http://github.com/ponyatov/calc/releases/latest}{prebuild windows
binary}] for first try.
\end{description}

\section{skelex: lexical program project skeleton}

First, we'll see how to organize our tiny project.
\bigskip

Nowdays you use huge IDE for
software development, but I prefer more light, portable and easy way: I use
(g)vim \ref{vim} text editor and Makefile\note{and Eclipse for more complex
cases}. Vim has strange key bindings, and can be some cryptic for a newbie, but
is very light in resources and have useful syntax coloring customization described in
details in vim syntax coloring \ref{vimcolor}.

\begin{tabular}{l l l}
src.src & script & sample source code \\
log.log & & execution log \\
ypp.ypp & yacc & syntax parser \\
lpp.lpp & lex & lexer using regexps \\
hpp.hpp & \cpp & headers \\
cpp.cpp & \cpp & runtime system we are going to implement \\
Makefile & make & project build script \\
rc.rc & linux & (g)vim start helper \\
bat.bat & windows & (g)vim start helper \\
.gitignore & git & ignored file masks \\
ftdetect.vim & vim & file type detection \\
syntax.vim & vim & syntax coloring \ref{vimcolor} for custom script \\
\end{tabular}

\subsection{Makefile: build script}

For project build, you need to \emph{track file interdependency} and do some
actions \emph{only on changed files}. So we can describe out dependency/action
rules in tiny Makefile, and run make tool by hotkey in editor every time we need
to compile or run project. Consult Addendum: GNU Make \ref{make} for details,
here we see only tiny Makefile snippet.

\clearpage
\begin{lstlisting}
log.log: src.src ./exe.exe
    ./exe.exe < $< > $@ && tail $(TAIL) $@
C = cpp.cpp ypp.tab.cpp lex.yy.c
H = hpp.hpp ypp.tab.hpp
L = -lreadline
./exe.exe: $(C) $(H)
    $(CXX) $(CXXFLAGS) -o $@ $(C) $(L)
ypp.tab.cpp ypp.tab.hpp: ypp.ypp
    bison $<
lex.yy.c: lpp.lpp
    flex $<
\end{lstlisting}

\section{lex: lexer generator}

\section{yacc: parser generator}
