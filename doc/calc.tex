\chapter{Parsing in \cpp: simple calculator}\clearpage

If you don't interested in low-level programming in \cpp, please skip this
chapter. But if you know C++ a bit, look at this theme much closer: flex/bison
is cool tools lets you do lot of things looks very complex on first glance:
\begin{description}
\item[process any data in text format], like config files and data for your
programs, it is very suitable to have this data in human readable form
\item[process complex command line]: pack \verb|argv[]| into one string and
interpret it as script
\item[process source code], you can make syntax colorer writes into .html files,
in few lines, using only flex.
\end{description}

In this chapter, we'll see how to implement simple calculator \textit{with infix
syntax} and variables, works in console. It is a quite useful program,
especially if your job coupled with engineering or science. I myself constantly
use it making some CADding and in occasionally everyday use.

In next sections, we'll see how to add some very complex in fact theme:
\emph{user-defined functions}, some control constructions, and arrays.

\bigskip
You can download full source code from separate github repo:
\url{http://github.com/ponyatov/calc}, and
\href{http://github.com/ponyatov/calc/releases/latest}{prebuild windows
binary} for first try.

\section{skelex: lexical program project skeleton}

First, we'll see how to organize our tiny project.
\bigskip

Nowdays you use huge IDE for software development, but I prefer more light,
portable and easy way: I use (g)vim \ref{vim} text editor and Makefile\note{and
Eclipse for more complex cases}. Vim has strange key bindings, and can be some
cryptic for a newbie, but is very light in resources and have useful syntax
coloring customization described in details in vim syntax coloring
\ref{vimcolor}.

\bigskip
\begin{tabular}{l l l}
src.src & script & sample source code \\
log.log & & execution log \\
ypp.ypp & yacc & syntax parser \\
lpp.lpp & lex & lexer using regexps \\
hpp.hpp & \cpp & headers \\
cpp.cpp & \cpp & runtime system we are going to implement \\
Makefile & make & project build script \\
rc.rc & linux & (g)vim start helper \\
bat.bat & windows & (g)vim start helper \\
.gitignore & git & ignored file masks \\
ftdetect.vim & vim & file type detection \\
syntax.vim & vim & syntax coloring \ref{vimcolor} for custom script \\
\end{tabular}

\clearpage
\begin{lstlisting}[title=rc.rc]
#!/bin/sh
gvim -p src.src log.log \
				ypp.ypp lpp.lpp hpp.hpp cpp.cpp Makefile
\end{lstlisting}

\subsection{Makefile: build script}

For project build, you need to \emph{track file interdependency} and do some
actions \emph{only on changed files}. So we can describe out dependency/action
rules in tiny Makefile, and run make tool by hotkey in editor every time we need
to compile or run project. Consult Addendum: GNU Make \ref{make} for details,
here we see only tiny Makefile snippet.

\clearpage
\lstinputlisting[title=Makefile,language=make]{../calc/Makefile}

\section{lex: lexer generator}

Lexer and parser files use same header with \verb|#include <headers>|:
\begin{lstlisting}[title={lpp.lpp}]
%{
#include "hpp.hpp"
%}
\end{lstlisting}
Lexer file \emph{must end with empty line}, don't forget to place EOL in last
string.
\begin{lstlisting}[title={ypp.ypp}]
%{
#include "hpp.hpp"
%}
\end{lstlisting}

\clearpage
Another \emph{required} section is rules, but for fist time it can be empty:
\begin{lstlisting}[title={lpp.lpp}]
%%
\end{lstlisting}

\noindent
Now you can run flex, and get resulting generated lexer source in lex.yy.c file:
\begin{lstlisting}
$> flex lpp.lpp && ls -la lex*
-rw-r--r-- 1 ponyatov ponyatov 43935 nov 20 19:43 lex.yy.c
\end{lstlisting}
And we have a problem: there is no \verb|lex.yy.h| header file, contains
\verb|yylex()| and \verb|yy_scan_string()| function declaration we required to
parse every string, entered in interactive command line\note{using readline
library}. To fix it, we must add option will create header file for as.

\section{yacc: parser generator}
