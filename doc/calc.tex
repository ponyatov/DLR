\chapter{Parsing in \cpp: simple calculator}\clearpage

If you don't interested in low-level programming in \cpp, please skip this
chapter.
\bigskip

In this chapter, we'll see how to implement simple calculator \textit{with infix
syntax} and variables, works in console. It is a quite useful program,
especially if your job coupled with engineering or science. I myself constantly
use it making some CADding and in occasionally everyday use.

In next sections, we'll see how to add some very complex in fact theme:
\emph{user-defined functions}, some control constructions, and arrays.

\bigskip
You can download
\begin{description}
\item[full source code]\ \\ from separate github repo:
\url{http://github.com/ponyatov/calc}, and
\item[\href{http://github.com/ponyatov/calc/releases/latest}{prebuild windows
binary}] for first try.
\end{description}

\section{skelex: lexical program skeleton}

First, we'll see how to organize our tiny project.
\bigskip

Nowdays you use huge IDE for
software development, but I prefer more light, portable and easy way: I use
(g)vim \ref{vim} text editor and Makefile\note{and Eclipse for more complex
cases}. Vim has strange key bindings, and can be some cryptic for a newbie, but
is very light in resources and have useful syntax coloring customization described in
details in vim syntax coloring \ref{vimcolor}.



\subsection{Makefile: build script}

\section{lex: lexer generator}

\section{yacc: parser generator}
