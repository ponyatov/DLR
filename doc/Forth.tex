\part{Something different: FORTH}\label{forth}

If you work with really small computer systems, like custom hardware build on
Cortex-M microcontroillers, \textit{you have very small amount of RAM}. The
topmost microcontroller family STM32F7 MCU used in
\href{http://www.st.com/en/evaluation-tools/32f746gdiscovery.html}{STM32F7GDISCOVERY}
board has \emph{only 340K of RAM}.

For this narrow case we have FORTH well known from end of 70s, and its big
brother OpenFirmware. In this part we'll see how we can implement tiny FORTH
system using bytecode approach.

\bigskip
In \verb|FORTH/| subdirectory you can see sources of bytecode compiler and
virtual machine (bytecode interpreter), with assember written in
flex/bison\note{it can be interesting for you how to implement tiny assembler,
without problems caused by concrete machine language details\ --- bytecode
simple commands format is very easy to understand}.

\section{FORTH/ file structure}

\begin{tabular}{l l l}
src.src & assembly-like & FORTH system source code \\
& syntax &\\
log.log & & logged execution of \\&&VM running compiled system \\
ypp.ypp & bison & syntax parser \\
lpp.lpp & flex & token lexer \\
hpp.hpp & \cpp & headers \\
cpp.cpp & \cpp & compiler elements and virtual machine \\
Makefile & make & build script\\&&(can be sample for any program uses
flex/bison)\\
FVM.exe & executable & assembler and virtual machine bundle \\
bin.bin & bytecode & compiled FORTH system bytecode\\&& dumped after VM
execution
\\
\end{tabular}
