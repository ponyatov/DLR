\part{Something different: flat memory FORTH}\label{forth}

As we see in \ref{pyforth}, it is easy to write some \F-like system in few
hours in any programming language. Manipualting with objects on stack is simple,
and let you use any side libraries and language high level features. 
But if you work with really small computer systems, like custom hardware build
on Cortex-M microcontroillers, \textit{you have very small amount of RAM}. The
topmost microcontroller family STM32F7 MCU used in
\href{http://www.st.com/en/evaluation-tools/32f746gdiscovery.html}{STM32F7GDISCOVERY}
board has \emph{only 340K of RAM}. \textcolor{green}{There is no space to run
fat dynamic language runtime with OOP}.
For this narrow case we have \F\ well known from the end of 70s, and its big
brother OpenFirmware. In this part we'll see how we can implement tiny
\F\ system using bytecode approach\note{It can be interesting for you how to
implement tiny assembler, without problems caused by concrete machine language
details\ --- bytecode simple commands format is very easy to understand.}\ but
low memory.

% \chapter{FVM: FORTH Virtual Machine}
% \bigskip
In \verb|FORTH/| subdirectory you can see sources of bytecode compiler and
virtual machine (bytecode interpreter), with assember written in
flex/bison.

\section{\F\ file structure}

\begin{tabular}{l l l}
src.src & assembly-like & \F\ system source code \\
& syntax &\\
log.log & & logged execution of \\&&VM running compiled system \\
ypp.ypp & bison & syntax parser \\
lpp.lpp & flex & token lexer \\
hpp.hpp & \cpp & headers \\
cpp.cpp & \cpp & compiler elements and virtual machine \\
Makefile & make & build script\\&&(can be sample for any program uses
flex/bison)\\
FVM.exe & executable & assembler and virtual machine bundle \\
bin.bin & bytecode & compiled \F\ system bytecode\\
&& dumped after VM execution\\
\end{tabular}

\section{Virtual Machine Architecture}

FVM\note{\F\ Virtual Machine}\ has one byte-addressed memory, and two separate
stacks:
\begin{description}[nosep]
\item[data stack] for data 
\item[return stack] return addresses for call/ret commands
\end{description}

\bigskip
Sizes of this structures was defined by constants, but you can modify code and
use expandable storage type like \verb|vector|\note{it will be much slower}

\begin{lstlisting}[language=C]
#define Msz 0x1000		/* bytes */
#define Rsz 0x100
#define Dsz 0x10 
\end{lstlisting}

\clearpage\noindent
\F\ has special \verb|CELL| constant, corresponds to \textit{machine word size
in bytes}.

\begin{lstlisting}[language=C]
#define CELL sizeof(int32_t)
\end{lstlisting}

\subsection{Memory}

\begin{lstlisting}[language=C]
extern uint8_t  M[Msz];	// memory
extern uint32_t Ip=0;	// instruction pointer
extern uint32_t Cp=0;	// compilation pointer (free heap)
\end{lstlisting}

Main memory contains all:
\begin{itemize}[nosep]
  \item compiled bytecode
  \item vocabulary structure \ref{vocabulary}
  \item data (constants, variables, strings, binary blobs,\ldots)
  \item heap from current \verb|Cp| till end of \verb|M[]| \ref{Fheap}
\end{itemize}

Memory has byte adressing, so we need some functions to get/set CELLs:

\begin{lstlisting}[language=C]
extern void set(uint32_t addr, int32_t value);
extern uint32_t get(uint32_t addr);
\end{lstlisting}

If you set \verb|cell < 0x100| , but read byte on same address, you must get the
same value. On \term{little-endian} machines (x86) we can read/write cells using
\verb|(uint32_t*)&M[addr]| pointer, but for portability we use this
byte-shifting functions:

\begin{lstlisting}[language=C++]
void set(uint32_t addr, int32_t value) {
	assert(addr+3 < Msz);			  // check memory bound
	M[addr+0] = (value>>0x00) & 0xFF;
	M[addr+1] = (value>>0x08) & 0xFF;
	M[addr+2] = (value>>0x10) & 0xFF;
	M[addr+3] = (value>>0x18) & 0xFF;	}
\end{lstlisting}
\begin{lstlisting}[language=C++]
uint32_t get(uint32_t addr) {
	assert(addr+3 < Msz);
	return \
		M[addr+0]<<0x00 | M[addr+1]<<0x08 | \
		M[addr+2]<<0x10 | M[addr+3]<<0x18;	}
\end{lstlisting}

\subsection{Compilation (in terms of \F)}

\begin{lstlisting}[language=C++]
extern uint32_t Cp;		// compilation pointer (free heap)
\end{lstlisting}

In \F\ term \term{compilation} means \textit{adding bytes to the end of
vocabulary}, in fact into the begin of a heap, moving heap bottom to higher
addresses. In \F\ standard there is only \verb|HERE| word returns address of
the heap begin (it must be \verb|HEAP| name definitely). So to address we'll use
special \verb|Cp| register.\label{Fheap}

\begin{lstlisting}[language=C++]
extern void Cbyte( uint8_t);	// compile byte
extern void Ccell(uint32_t);	// compile cell
extern void Cstring(char*);		// compile ASCIIZ string
\end{lstlisting}

\noindent
These functions will be used in assembler.
\bigskip

\label{Cxxx}
\begin{lstlisting}[language=C++]
void Cbyte( uint8_t b) {
	M[Cp++] = b; assert(Cp<Msz); }
void Ccell(uint32_t c) {
	set(Cp,c); Cp+= CELL; assert(Cp<Msz); }
void Cstring(char* s) {
	uint32_t L = strlen(s); assert(Cp+L+1<Msz);	// length
	memcpy(&M[Cp],s,L+1); Cp += L+1; }	// compile length+1
\end{lstlisting}

\subsection{Vocabulary structure}\label{vocabulary}

In \F\ terms \term{word} means some active data element, analogous to function
and procedure in mainstream languages. Variables and constants in \F\ also
words. It corresponds to \term{word} in human languages\ --- sequence of
letters, which means something. When you enter some \F\ code in command line,
interpreter \ref{INTERPRET} searches each word\note{delimited by space symbols}\
in \term{vocabulary}, and executes \ref{EXECUTE} it if search was successful.

\bigskip
\term{Vocabulary} is container data structure, implements:
\begin{description}[nosep]
\item[words storage] in linked list order\note{or tree of linked lists in case
of multiple vocabulary supported}
\item[word search] by its name
\item[definition] of new words using compiling words (see \verb|Cxxx()|
functions \ref{Cxxx})
\end{description}

Every item in vocabulary has this fields structure\note{If you plan to do some
hacking using bytecode for software writing, you can eliminate vocabulary
headers in case of you do not use vocabulary search. To do this, you can
fork your own assembler, and remove all calls in Cheader() except CFA(). CFA
compilation is required because it sets \_entry field in first jmp command to
last defined word (see next page).}:

\bigskip\noindent
\begin{tabular}{l l l l}
LFA & cell & Link Field Area & link to previous word or 0 \\
AFA & byte & Attribute Field Area & flags, IMMED \ref{IMMEDIATE} \\
NFA & asciiz string & Name Field Area & word name \\
CFA & bytecode & Code Field Area & executable bytecode \\
PFA & optional & Parameters Field Area & in variables and constants \\
\end{tabular}
\clearpage

\noindent
Last defined word must be marked somewhere
\begin{itemize}[nosep]
  \item 
as entry point on system start, and
  \item 
as first point in search and compilation, 
\end{itemize}
so we need special fields in the beginning of memory image:

\begin{lstlisting}[language=C++]
// program entry point (addr of jmp parameter)
#define _entry  1
// last defined word LFA address
#define _latest (_entry+CELL)

int main(int argc, char *argv[]) {
	// ============ compile vocabulary header
	// jmp _entry	jump to last defined word
	Cbyte(opJMP); Ccell(0);
	// _latest		LFA of last defined word
	Ccell(0);
\end{lstlisting}

To compile vocabular header use
\begin{lstlisting}[language=C++]
map<string,uint32_t> SymTable;				// symbol table

void LFA() {
	uint32_t L = get(_latest); set(_latest,Cp); Ccell(L); }
void AFA(uint8_t b) { 
	Cbyte(b); }
void NFA(char* s) { 
	Cstring(s); }
void CFA(string name) { 
	SymTable[name] = Cp; set(_entry,Cp); }
void Cheader(char* name) {				  // compile header
	LFA(); AFA(); NFA(name); CFA(name); }
\end{lstlisting}

\subsection{Bytecode interpreter}

Bytecode interpreter will be run after assembler ended its work:

\begin{lstlisting}[language=C++]
int main() {
	...						// compile vocabulary header
	yyparse();				// run compiler
	dump();					// dump memory into .bin file
	VM();					// run VM
}	
\end{lstlisting}

As any other computer, interpreter implements fetch/decode/execute loop over
commands in \verb|M[]| memory, with command pointed by \verb|Ip| instruction
pointer register.

\begin{lstlisting}[language=C++]
void VM() { for (;;) {						// infty loop
	printf("%.4X ",Ip);
	uint8_t op = M[Ip++]; assert(Ip<=Cp);		 // FETCH
	printf("%.2X ",op);
	switch (op) {						// DECODE/EXECUTE
		case opNOP : nop();  break;
		case opBYE : bye();  break;
		case opJMP : jmp();  break;
		case opCALL: call(); break;
		case opRET : ret();  break;
		case opLIT : lit();  break;
		default:
			printf("bad opcode\n\n"); abort();
	}
	printf("\n");
}}
\end{lstlisting}

\section{Core command set}

FVM uses two bytecode command types:
\begin{description}[nosep]
\item[CMD0] one byte opcode without parameters
\item[CMD1] byte opcode with required cell-sized parameter
\end{description}

\begin{lstlisting}[title=ypp.ypp: yacc syntax parser]
%defines %union { char *s; uint8_t op; uint32_t n; }
%token <op> CMD0 CMD1
\end{lstlisting}

\subsection{Control flow}

\begin{lstlisting}[language=C++]
#define opNOP	0x00	// nop
#define opBYE	0xFF	// bye
#define opJMP	0x01	// jmp <addr>
#define opQJMP	0x02	// ?jmp <addr>
#define opCALL	0x03	// call <addr>
#define opRET	0x04	// ret
#define opLIT	0x05	// lit <value>
\end{lstlisting}

\begin{description}

\item[nop] do nothing
\begin{lstlisting}[language=C++]
#define opNOP	0x00	// nop
\end{lstlisting}
\begin{lstlisting}[language=C++]
void nop() { printf("nop"); }
\end{lstlisting}

\end{description}

\subsection{Stack manipulations}
\subsection{Arithmetics and binary operations}
\subsection{Basic input/output (console, files)}
\subsection{Fast Foreign Interface}

\section{Extensions}
\subsection{Native GUI}
\subsection{Networking}
\subsection{Database connection}

\section{Media and gaming}
\subsection{Simple Direct Layer}
\subsection{OpenGL}
\subsection{Media codecs}

\section{CAD/CAD/CAE and numerical math}
\subsection{Vector schematics}
\subsection{EDA Electronics Design}
\subsection{Numerical math}
\subsection{Vizualization}

\section{Compiler framework}
\subsection{Parser generator}
\subsection{LLVM}
