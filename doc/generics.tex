\chapter{Generics: data types and algorithms}\clearpage

As \emph{we are going to use \term{code autogeneration}}, we must create type
system able to mimic any programming language data model, and \term{autogen}
code specific to typically use cases. Another important thing: we must implement
large set of widely known generic algorithms, and let them to be used as
first class dynamic script elements (as algorithm libraries).
To implement this we can define special classes set, powered with
\term{reflection} and \emph{dynamic behaviour}:
\begin{description}[nosep]
\item[Primitive] superclass for elements can be implemented using
computer hardware capabilities at low level code, like numbers and strings;
\item[Collection] is anything can contain another data elements inside themself
\item[AST] as widely accepterd form of programming languages constructs, as
we are going to wors with program elements as is
\item[Active] elements represents active part of computing system
\item[Function] and methods is widely used as main programming language
construct
\item[Algorithm] would be great to be able to manipulate as first class object
\item[Thread] is single process of sequential computing required for
multitasking
\end{description}

\fig{../tmp/sym_00.pdf}{width=\textwidth}%height=\textheight}%

\fig{../tmp/sym_01.pdf}{width=\textwidth}%height=\textheight}%
