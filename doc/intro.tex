\clearpage
\section{Intro}%\addcontentsline{toc}{section}{Intro}

I spent a lot of time asking about some ready library or framework let me easy
transform simple syntax parser written in flex/bison into dynamic language
interpreter. Search still in progress, teraton runtimes like JVM not preferred,
portability and light resource requirements in focus.

\bigskip
So I'm dreaming about some mix of:
\begin{itemize}[nosep]
  \item \emph{Python}ic syntax 
  \item \emph{SmallTalk} live object system\note{able to run on
  Beowulf-like SSI cluster networks thanks to message passing scalability}\ and
  hibernation
  \item \emph{Erlang} parallelism and stability
  \item object/graph database engine embedded, RDBMS connectivity
  \item extremal portability much more then \emph{Java} (Android in first place)
  \item tiny RAM usage, light and fast run, one executable file installation
\end{itemize} 

\section{About this manual}%\addcontentsline{toc}{section}{About this manual}

This manual consist of this parts:
\begin{description}
\item[Part \ref{newbie}] Tutorial\\
If you just start to learn about implementing programming languages, and want to
write your own script language, it is better to start from a simple. In this
manual part, we kick out dirty details of full-featured DLR system, and look at
an \emph{working model} as skeleton of full system, written in Python.
\item[Part \ref{reference}] Reference\\reference manual for core system
\item[Part \ref{forth}] Something different: FORTH\\
for really small systems: implementing for Cortex-M (STM32F7) 
\end{description}
