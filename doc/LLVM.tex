\chapter{Using LLVM}\clearpage

The \href{http://llvm.org/}{LLVM Project}\note{Low Level Virtual Machine} is a
collection of modular and reusable compiler and toolchain technologies. Despite
its name, LLVM has little to do with traditional virtual machines. Project goal
was providing a modern, SSA\note{In compiler design, static single assignment
form (often abbreviated as SSA form or simply SSA) is a property of an
intermediate representation (IR), which requires that each variable is assigned
exactly once, and every variable is defined before it is used.}-based \ref{SSA}
compilation strategy capable of supporting both static and dynamic compilation
of arbitrary programming languages. The primary sub-projects of LLVM are:
\begin{description}
\item[LLVM Core] libraries provide a modern source- and target-independent
\emph{optimizer}, along with \emph{code generation} support for many popular
CPUs
\item[Clang] is C/\cpp/Objective-C compiler
\item[LLDB] builds on libraries provided by LLVM and Clang to provide a
\emph{debugger}.
\end{description}

\bigskip
\href{http://www.llvmpy.org/}{llvmpy} is a Python wrapper around the LLVM \cpp\
library which allows simple access to compiler tools.
It can be used for a lot of things, but here are some ideas:
\begin{itemize}[nosep]
  \item 
dynamically create LLVM IR for linking with LLVM IR produced by \verb|CLANG| or
\verb|dragonegg|
  \item 
build machine code dynamically using LLVM execution engine
  \item 
use together with PLY or other tokenizer and parser to write a complete compiler
in \py
\end{itemize}

\section{Installation}

\begin{verbatim}
$ sudo apt install python-llvm python-ply
or
$ sudo pip install llvmpy ply
\end{verbatim}
\lstinputlisting[title=llvm\_installed\_test.py,language=Python]{llvm01.py}

Tiny executable module does nothing, done using
\href{https://eli.thegreenplace.net/2012/08/10/building-and-using-llvmpy-a-basic-example}{this manual}:
\lstinputlisting[title=llvm\_null\_module.py,language=Python]{llvm02.py}

\section{SSA: Single State Assignment}\label{SSA}

