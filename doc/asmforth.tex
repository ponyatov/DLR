\section{\F: Command Shell}

Program works, but we need something more interesting. Lets do some
\emph{interactive system}, in parralel \emph{adding some control constructions}
to our assembler. At this point we
\begin{itemize}
	\item have no tools let as define complex syntax in our language
	\item we dont want to mess two parsers: assembler and command shell
\end{itemize}
So we need something very strange for our command shell: programming language
without syntax. In fact it is impossible, but there is one language with ultra
simple syntax which \emph{parser can be rewritten in few machine commands}: \F.

If you know something about \F, it is not suprize for you that we use it: our
\term{virtual machine}
was designed very close to this language principles. We do some shift
from original \F, as we want to manipulate with objects, but not with bytes
and machine integers (see next page), as you do making original
\term{Virtual FORTH Machine}.

\bigskip
Returning to \F\ syntax, it is very simple:
\begin{enumerate}[nosep]
\item collect sequential symbols one by one from input stream until first
space encountered\ --- \emph{tadaam! thats all: you have got the parser 
finished 8-)}
\item and try to do something with selected string (not surprizing that this 
string has special name in \F\ --- \term{word}).
\end{enumerate}
On next stage \F-system tries to find this word in special structure\ ---
\term{vocabulary}. Vocabulary is linked list contains compiled procedures
with its names stored in.
If parsed word matches procedure name, this procedure will
be executed immediately. If word not found in vocabulary, \F\ tries to convert
it into integer number\note{there is no floating point support in core
\F\ language\ --- are you already scared?}, and push into \term{data stack}.

\bigskip
We are not going to write yet another tutorial on \F\ here. It is better
if you take a break here, and look into first chapters of real \emph{cool book:
Leo Brodie's Starting \F}\ \cite{starting}. You will be amazed by \F\ miracle
simplicity, and scared by its dragon ass \cite{dragon}\ of zen syntax,
data stack bitbanging, and functional limitations\note{no float numbers,
no files, do you want data structures? make arrays yourself!}.

\bigskip
Also we will do some ANS Forth standard sidestep, see notes.

\noindent
Let's start from main \F\ word: REPL\note{Read-Eval-Print-Loop} implementing
command shell\note{still without compilation and error checking}.
\begin{lstlisting}[language=Forth]
: INTERPRET		\ REPL interpreter loop
	begin
		\ get next word name from input stream
		word	( -- str:wordname ) 
		\ find word entry point
		find	( str:wordname -- addr:xt )
		\ execute found xt (execution token) 
		execute	( xt -- ) 
	again
;
\end{lstlisting}
Don't forget how we should pass this source for compilation and run:
enclose it into \verb|VM(''' ... code ... ''')| in \py\ source.

\subsection{Expanding VM using inheritance}

A current parser will not process this source, as syntax notably changed.
We will not break VM demo parser, but use \emph{VM class inheritance to redefine
parser} for special \F\ dialect, and add some extra command to original VM.
\begin{lstlisting}[language=Python]
		t_ignore_COMMENT = r'\#.+|\(.*?\)'		# comments
		# token types
		tokens = ['REGISTER','EQ','STRING',
			# parse all VM commands with low case letters
			'COMMAND',								
			# control constructs
			'COLON','SEMICOLON','BEGIN','AGAIN']	
\end{lstlisting}
