\section{\F: Command Shell}

Program works, but we need something more interesting. Lets do some
\emph{interactive system}, in parralel \emph{adding some control constructions}
to our assembler. At this point we
\begin{itemize}
	\item have no tools let as define complex syntax in our language
	\item we dont want to mess two parsers: assembler and command shell
\end{itemize}
So we need something very strange for our command shell: programming language
without syntax. In fact it is impossible, but there is one language with ultra
simple syntax which \emph{parser can be rewritten in few machine commands}: \F.

If you know something about \F, it is not suprize for you that we use it: our
\term{virtual machine}
was designed very close to this language principles. We do some shift
from original \F, as we want to manipulate with objects, but not with bytes
and machine integers (see next page), as you do making original
\term{Virtual FORTH Machine}.

\bigskip
Returning to \F\ syntax, it is very simple:
\begin{enumerate}[nosep]
\item collect sequential symbols one by one from input stream until first
space encountered\ --- \emph{tadaam! thats all: you have got the parser 
finished 8-)}
\item and try to do something with selected string (not surprizing that this 
string has special name in \F\ --- \term{word}).
\end{enumerate}
On next stage \F-system tries to find this word in special structure\ ---
\term{vocabulary}. Vocabulary is linked list contains compiled procedures
with its names stored in.
If parsed word matches procedure name, this procedure will
be executed immediately. If word not found in vocabulary, \F\ tries to convert
it into integer number\note{there is no floating point support in core
\F\ language\ --- are you already scared?}, and push into \term{data stack}.

\bigskip
We are not going to write yet another tutorial on \F\ here. It is better
if you take a break here, and look into first chapters of real \emph{cool book:
Leo Brodie's Starting \F}\ \cite{starting}. You will be amazed by \F\ miracle
simplicity, and scared by its dragon ass \cite{dragon}\ of zen syntax,
data stack bitbanging, and functional limitations\note{no float numbers,
no files, do you want data structures? make arrays yourself!}.

\bigskip
Also we will do some ANS Forth standard sidestep, see notes.

\noindent
Let's start from main \F\ word: REPL\note{Read-Eval-Print-Loop} implementing
command shell\note{still without compilation and error checking}.
\begin{lstlisting}[language=Forth]
: INTERPRET		\ REPL interpreter loop
	begin
		\ get next word name from input stream
		word	( -- str:wordname ) 
		\ find word entry point
		find	( str:wordname -- addr:xt )
		\ execute found xt (execution token) 
		execute	( xt -- ) 
	again
;
\end{lstlisting}
Don't forget how we should pass this source for compilation and run:
enclose it into \verb|VM(''' ... code ... ''')| in \py\ source.

\subsection{Expanding VM using inheritance}

A current parser will not process this source, as syntax notably changed.
We will not break VM demo parser, but use \emph{VM class inheritance to redefine
parser} for special \F\ dialect, and add some extra command to original VM.

When you are going to implement your own language on top of VM, you should 
go this way too:
\begin{itemize}
\item fork special class from core VM
\item implement parser/compiler for your language
\item \emph{expand inherited VM} with specific commands your need of
\end{itemize}
Last item must be focused: \emph{don't change core VM class} with new commands
or modified behaviour, as this changes will impact on all other inherited VMs.
We broke this rule here, only because VM can't do anything in this point, and we
need a lot of to add to make core VM be usable itself.

\begin{lstlisting}[language=Python]
# it should look like this:
class FORTH(VM):
	t_ignore_COMMENT = r'\#.*|\\.*|\(.*?\)'		# comment
\end{lstlisting}

And here we have a problem, Houston! All elements of parser are encapsulated
in \verb|compiler()| method using closure, and oops we \emph{must rework VM
class} itself. The way we can do it you can see on this short sample. This is
lexer-only variant of simplest Hello World parser uses inheritance:
\lstinputlisting[language=python]{../inher.py}

I faced with problem in PLY: its lexer generator have problem with
\verb|t_error| unicality validation, if there are more \verb|t_error| functions
(or methods) defined anywhere in one module. So I made tiny fix in
\verb|ply/lex.py|, see
\href{http://github.com/dabeaz/ply/issues/142}{issue 142}, but my pull request
still not merged, and \emph{you must clone patched PLY right here}:
\begin{lstlisting}
~/DLR$ git clone -o ponyatov \
	https://github.com/ponyatov/ply.git
\end{lstlisting}
For import patched PLY we need to push it in search path \emph{before}
\verb|/usr/lib|:
\begin{lstlisting}[language=Python]
import sys ; sys.path.insert(0,'./ply')
import ply.lex  as lex ; ply_class_inherit = True # use fix
\end{lstlisting}

\bigskip
Going to rewrite VM, we need \verb|t_rules| as class static strings, and
methods, moved out of \verb|compiler()|:
\begin{lstlisting}[language=Python]
class VM:
	# ===== lexer code section =====
	t_ignore = ' \t\r'				# drop spaces (no EOL)
	t_ignore_COMMENT = r'\#.*'		# line comment
	# regexp/action rules (ANY state)
	def t_ANY_newline(self,t):		# special rule for EOL
		r'\n'
		t.lexer.lineno += 1			# increment line counter
		# do not return token, it will be ignored by parser
	# token types
	tokens = ['NOP','BYE','REGISTER','EQ','STRING']
	# required lexer error callback must be method
	def t_ANY_error(self,t): raise SyntaxError('lexer: %s' % t)
\end{lstlisting}
Expand comment regexp and tokens set in FORTH:
\begin{lstlisting}[language=Python]
class FORTH(VM):
	t_ignore_COMMENT = r'\#.*|\\.*|\(.*?\)'		# comment
	tokens = ['NOP','BYE','REGISTER','EQ','STRING',
			'COLON','SEMICOLON','BEGIN','AGAIN']	# : ;
\end{lstlisting}
Change lexer creation:
\begin{lstlisting}[language=Python]
class VM:
	def compiler(self,src):
		...
		# create lexer object
		# note: we point on object (instance!) with rules 
		self.lexer = lex.lex(object=self)
		# feed source code
		self.lexer.input(src)
\end{lstlisting}
and parser the same:
\begin{lstlisting}[language=Python]
class VM:
	def compiler(self,src):
		# create ply.yacc object, without extra files
		parser = yacc.yacc(debug=False,write_tables=None,\
			module=self) # here we must point on instance
		# parse source code using lexer
		parser.parse(src,self.lexer)
\end{lstlisting}

In \F\ syntax we have no ability to distinct between VM commands, defined in
vocabulary and yet undefined words using only regexp. So we need to group tokens
in one ID universal token, and do some lookup to detect what token type we have
in every point.

\begin{lstlisting}[language=Python]
class FORTH(VM):
	tokens = ['ID','COLON','SEMICOLON','BEGIN','AGAIN']
\end{lstlisting}
We removed some token types, and PLY will give as report\note{PLY was developed
by David M. Beazley specially for student learning purposes, so good verbosity
is key feature of this parser generator. As we are going to build power
\emph{metaprogramming system}, which can be thinked as \textit{interactive
compiler \& algorithm development environment}, we should go this way too
later making parser generators and other tools, visualization and verbosity must
have.}\ what's wrong:
\begin{lstlisting}
ERROR: /home/ponyatov/DLR/VM.py:62:
	Symbol 'NOP' used, but not defined as a token or a rule
ERROR: /home/ponyatov/DLR/VM.py:65:
	Symbol 'BYE' used, but not defined as a token or a rule
ERROR: /home/ponyatov/DLR/VM.py:68:
	Symbol 'REGISTER' used, but not defined as a token or a rule
ERROR: /home/ponyatov/DLR/VM.py:68:
	Symbol 'EQ' used, but not defined as a token or a rule
ERROR: /home/ponyatov/DLR/VM.py:73:
	Symbol 'STRING' used, but not defined as a token or a rule
WARNING: Token 'AGAIN' defined, but not used
WARNING: Token 'BEGIN' defined, but not used
WARNING: Token 'SEMICOLON' defined, but not used
WARNING: Token 'COLON' defined, but not used
WARNING: Token 'ID' defined, but not used
WARNING: There are 5 unused tokens
ERROR: Infinite recursion detected for symbol 'constant'
ERROR: Infinite recursion detected for symbol 'command'
ply.yacc.YaccError: Unable to build parser
\end{lstlisting}
Here we see source line numbers, points directly on problematic gramma rules,
and list of unused tokens.

So we need to drop corresponding rules
from inherited class.
The second tiny patch i've done for PLY, lets mark parser rules as deleted using
\verb|=Null| assignment:
\begin{lstlisting}[language=Python]
class FORTH(VM):
	t_NOP = p_command_NOP = None
 	t_BYE = p_command_BYE = None
 	t_REGISTER = p_command_R_load
 	p_constant_STRING = None
\end{lstlisting}
\begin{lstlisting}
ERROR: /home/ponyatov/DLR/VM.py:60: Symbol 'command' used,
	but not defined as a token or a rule
\end{lstlisting}
We must define what commands we are going to see in our programs.
\begin{description}[nosep]
\item[grammar] rules modifies VM grammar, and required
\begin{lstlisting}[language=Python]
# grammar override
def p_command_ID(self,p):		' command : ID '
def p_command_COLON(self,p):	' command : COLON '
def p_command_SEMICOLON(self,p):' command : SEMICOLON '
def p_command_BEGIN(self,p):	' command : BEGIN '
def p_command_AGAIN(self,p):	' command : AGAIN '
\end{lstlisting}
\item[lexemes] we want to override comparing to base VM regexp rule set
\begin{lstlisting}[language=Python]
# lexemes regexp overrides
t_COLON = ':' ; t_SEMICOLON = ';'
def t_BEGIN(self,t):
	r'begin'
	return t 
def t_AGAIN(self,t):
	r'again'
	return t
def t_ID(self,t): # this rule must be last rule
	r'[a-zA-Z0-9_]+'
	return t 
\end{lstlisting}
\end{description}
AssertionError, we got empty program:
\verb|self.Ip < len(self.program)|

Let's add fix to begin of source lets run dummy program as usual:
\begin{lstlisting}[language=Python]
FORTH(r''' # use r' : we have escapes in string constants
nop bye
...
\end{lstlisting}
Base \verb|VM| compiler detects this commands directly in parser syntax rules,
and compiles them. In new inherited \verb|FORTH| machine we disabled this rules,
and moved all word-looking char sequences into one \verb|ID| token. In sample
source you can see that this ID can be new word name, VM command or control
structure statement\note{In \F\ all control structures must be implemented using
generic \F-words, we'll see how to make your own controls yourself later. But
now we will use \emph{parser-based method} to show you how to implement them in
your assembly or compiler}. In ID lex rule we can detect what is this ID in
real, and use \term{terminal overwrite}.

We have to add some hint let us check is given word name:
\begin{description}[nosep]
\item[VM command]\ \\
we can use \py\ \term{reflection} to look up is given string is FORTH
class method name or its VM superclass method, but for simplicity we will use
special dict:
\begin{lstlisting}[language=Python]
class FORTH(VM):
	# command lookup table: string -> method
	cmd = { 'nop':VM.nop , 'bye':VM.bye }
\end{lstlisting}
\item[already defined word] exists in vocabulary\\
it is intuitive we must have vocabulary itself:
\begin{lstlisting}[language=Python]
class FORTH(VM):
	# vocabulary of all defined words
	voc = {}
\end{lstlisting}
\item[undetected ID], maybe new word or some misspelling
\end{description}
Now we can add \term{lexeme overriding} into \verb|t_ID| lexer rule:
\begin{lstlisting}[language=Python]
	# add extra types for tokens
	tokens = ['ID','CMD','VOC',
				'COLON','SEMICOLON','BEGIN','AGAIN']
 	def t_ID(self,t): # this rule must be last rule
 		r'[a-zA-Z0-9_]+'
 		# first lookup in vocabulary
 		if t.value in self.voc: t.type='VOC'
 		# then check is it command name
 		if t.value in self.cmd: t.type='CMD'
 		return t 
\end{lstlisting}
Note checking order: this variant let as define new words with names equal to VM
commands, and later we'll do it in \term{\F-assembler} \ref{forthasm} let you
compile or execute single VM command in interactive \F\ system.
