\chapter{Terminology}\label{terminology}\clearpage

\section{Programming language vs Implementation}\label{implementing}

\begin{description}
\item[programming language] itself \emph{is only specification}\ \\
of syntax \ref{syntax} and semantics \ref{semantics}
\item[implementation] of programming language\ \\
is concrete program, build for specific targets \ref{target} set, with concrete
standard (runtime) library set shipped with
\item[reference implementation]\ \\
some widely known languages (CPython2, Ruby MRI) has some dominatic
implementation\note{written by language author: Guido van Rossum (\py),
Xerox (\ST), Yukihiro Matsumoto (Ruby)}, known as \term{reference
implementation}:
if you write your variant of language system, and want it to be compatible with
language you name, you can test it using ref implementation, every test must
behave identical.

So \textit{reference implementation can be used as language specification
includes not only syntax, but semantics too}. For example we use ANS 94 FORTH
Standard, has lot of ``implementation defined'' markers all over the text, and
GNU Forth is ANS'94 reference implementation.

\end{description}

\section{There is no compiled/interpreted language}

Programming language \emph{is formal specification} of syntax and semantics,
it is a documentation artifact, and this \emph{specification can't be
compiled}\note{in fact, it is not totally true\ --- this GNU DLR is in plans to
be such system, you specify some language using some formal language, compile
it using compiler compiler into executable code, and bind it with universal
runtime library gives your language implementation alive} !

For example, pure C is widely known as ``compiled \ref{compiler}\ language'',
but Fabrice Bellard's \href{http://bellard.org/tcc/}{Tiny C Compiler} can run
as interpreter (but uses dynamic compilation \ref{dynacomp}).

Java looks like classical compiler suite with edit/compile/debug
loop, but in fact \emph{JVM is interpreter} by design, \emph{expanded by JIT}
for programs speed up using dynamic compilation \ref{dynacomp}.

\section{Syntax and Semantics}

\begin{description}
\item[Syntax]\label{syntax} how it looks like, and how synatax elements relates
to each other in source code
\item[Semantics]\label{semantics} is what every construction and statement must
to do
\end{description}


\section{Compiler vs Interpreter}

\begin{description}
\item[Compiler]\ \\
is a program module (can be library, single program or large
software package) which compiles source code into some low-level code: machine
or bytecode

\setlength{\topsep}{0pt}
\begin{framed}
The key property: compiler \emph{compiles but not executes program}
\end{framed}
\item[Interpreter]\label{interpreter}\ \\
\begin{framed}
Interpreter is a computer program written in language L \emph{executes} program
written in language P.
\end{framed}

In other words, interpreter \emph{can execute program in machine code}, making
some \emph{real machine} \ref{VM} simulation. 
But machine code for real
processor ineffective in simulation\note{if you don't use
virtualization supported by hardware, like x86 code on modern CPUs like Intel
VT/VMX}, and especially ineffective in compiler code generation. That's why
\emph{all modern interpreters} use bytecode \ref{bytecode} for low-level program
representation.
\end{description}

\section{Bytecode vs Machine Code}

\begin{description}
\item[Machine code] is binary code for real CPU, must be executed by hardware\\
(but can be interpreted as well in special cases like simulators or tracing
debuggers)
\item[ByteCode]\label{bytecode} \textit{is a sort of machine code} for some
virtual machine \ref{VM}, \emph{specially tuned for interpretation}
\ref{interpreter}
\end{description}


\section{Virtual Machine}\label{VM}

\section{Static vs Dynamic}

\begin{description}
\item[Dynamic]
is all relates to things happening \emph{in runtime}
\begin{description}[nosep]
\item[dynamic compilation]\label{dynacomp} into real machine code can be used in
interpreters to recompile parts of programs changed \emph{in runtime}, or in
process of running tracking optimizer on real data processing
\item[dynamic language]\label{dynalang} is a programming language which have in
its specifiction \emph{ability to change parts of program in runtime}:
\begin{enumerate}[nosep]
  \item modify existing classes and object member set on the fly
  \item create new classes
  \item support code generation 
  \item process strings as source code, and compile and execute it
  \item use polymorphic functions which process parameters depends on its
  types
\end{enumerate} 
\end{description}
\item[Static]
\begin{description}[nosep]
  \item[compilation] in compile time only, program parts can't be changed and
  recompiled in runtime
  \item[class member] class element defined in class itself, not for derived
  object; \emph{every class member is shared between all class instances
  (objects)}
\end{description}
\end{description}
